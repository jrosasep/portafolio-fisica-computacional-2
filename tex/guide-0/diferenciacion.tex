\documentclass[../portafolio.tex]{subfiles}

\begin{document}

\chapter{Métodos de Diferenciación Numérica: Derivada Adelantada y Error Absoluto}
\label{ch:diferenciacion}
%%%%%%%%%%%%%%%%%%%%%%%%%%%%%%%%%%%%%%%%%%%%%%%%%%%%%%%%%%%%%%%%%%%%%%%%%%%%%%%%

\hfill \textbf{Fecha de la actividad:} 25 de noviembre de 2024

\medskip

En este capítulo, se analiza el esquema de diferencias adelantadas de tres puntos para aproximar la derivada de una función $f(x)$.

\section{Orden del Error de Truncamiento}
El esquema de diferencias adelantadas de tres puntos es:
\begin{equation}    \label{eq-dif:esq}
f'_{\text{num}}(x) = \frac{-3f(x) + 4f(x+h) - f(x+2h)}{2h}.
\end{equation}
Para determinar el orden del error de truncamiento entre este esquema y $f'(x)$, vamos a expandir $f(x+h)$ y $f(x+2h)$ en series de Taylor alrededor de $x$ hasta el quinto término.

Usamos las siguientes expansiones en series de Taylor:
\begin{equation}    \label{eq-dif:exp1}
f(x+h) = f(x) + f'(x)h + \frac{f''(x)}{2}h^2 + \frac{f^{(3)}(x)}{6}h^3 + \frac{f^{(4)}(x)}{24}h^4 + O(h^5),
\end{equation}
\begin{equation}    \label{eq-dif:exp2}
f(x+2h) = f(x) + 2f'(x)h + \frac{2^2 f''(x)}{2}h^2 + \frac{2^3 f^{(3)}(x)}{6}h^3 + \frac{2^4 f^{(4)}(x)}{24}h^4 + O(h^5).
\end{equation}

Sustituyendo estas expansiones \eqref{eq-dif:exp1} y \eqref{eq-dif:exp2} en el esquema de la derivada numérica \eqref{eq-dif:esq}, obtenemos:

\begin{equation*}
f'_{\text{num}}(x) = \frac{-3f(x) + 4\left( f(x) + f'(x)h + \frac{f''(x)}{2}h^2 + O(h^3) \right) - \left( f(x) + 2f'(x)h + 2f''(x)h^2 + O(h^3) \right)}{2h}.
\end{equation*}

Simplificando los términos, finalmente obtenemos:

\begin{equation}
f'_{\text{num}}(x) = f'(x) + O(h^3).
\end{equation}

Por lo tanto, el \textbf{orden del error de truncamiento} entre la derivada analítica $f'(x)$ y el esquema de derivada numérica \eqref{eq-dif:esq} es de $O(h^3)$.

\section{Estimación del Error Absoluto}
El error absoluto $E_{\text{abs}}$ entre la derivada exacta $f'(x)$ y la derivada numérica $f'_{\text{num}}(x)$ es dado por:
\begin{equation}
    E_{\text{abs}} = |f'(x) - f'_{\text{num}}(x)| \sim O(h^3) + \epsilon_{\text{redondeo}}.
\end{equation}

El error de redondeo $\epsilon_{\text{redondeo}}$ se refiere a la imprecisión que ocurre debido a las limitaciones de la representación numérica de los números en la computadora.

\section{Conclusiones}
El esquema de diferencias adelantadas de tres puntos proporciona una aproximación de orden $O(h^3)$ para la derivada de una función. Sin embargo, los errores de redondeo $\epsilon_{\text{redondeo}}$ deben ser considerados cuando se utiliza este método con \textbf{valores pequeños} de $h$.

\end{document}
