
\documentclass[../portafolio.tex]{subfiles}

\begin{document}

\chapter{Solución de Ecuaciones}
\label{ch:soluciones-ecuaciones}
%%%%%%%%%%%%%%%%%%%%%%%%%%%%%%%%%%%%%%%%%%%%%%%%%%%%%%%%%%%%%%%%%%%%%%%%%%%%%%%%
\hfill \textbf{Fecha de la actividad:} 27 de noviembre de 2024

\medskip
En esta sección, determinaremos numéricamente las soluciones de las siguientes ecuaciones:  
\begin{enumerate}
    \item \textbf{Ecuación 1}: $x^3 - 21x^2 + 120x - 100 = 0$
    \item \textbf{Ecuación 2}: $x = 2^{-x}$
    \item \textbf{Ecuación 3}: $\tan x = \frac{1}{x}$
\end{enumerate}
Para ello, implementaremos el método de Newton-Raphson, un procedimiento iterativo diseñado para encontrar las raíces de una función $f(x)$. A partir de un valor inicial o semilla $x_0$, la fórmula de recurrencia está dada por:  
\begin{equation}
x_{n+1} = x_n - \frac{f(x_n)}{f'(x_n)},
\end{equation}
donde $f'(x)$ es la derivada de $f(x)$. 

\textbf{Nota}: No estudiaremos la multiplicidad de las soluciones.
\section{Definimos funciones y sus derivadas para cada ecuación}
\begin{enumerate}
    \item \textbf{Ecuación 1}: $x^3 - 21x^2 + 120x - 100 = 0$.
    \begin{equation*}
        f(x)=x^3 - 21x^2 + 120x - 100
    \end{equation*}
    \begin{equation*}
        f'(x)=3x^2 - 42x + 120
    \end{equation*}
    
    \item \textbf{Ecuación 2}: $x = 2^{-x}$. Igualando a 0.
    \begin{equation*}
        g(x)=x-2^{-x}
    \end{equation*}
    \begin{equation*}
        g'(x)=1+\ln(x)\cdot2^{-x}
    \end{equation*}
    
    \item \textbf{Ecuación 3}: $\tan x = \frac{1}{x}$. Igualando a 0.
    \begin{equation*}
        h(x)=\tan x - \frac{1}{x}
    \end{equation*}
    \begin{equation*}
        h'(x)=\sec^2 x + \frac{1}{x^2}
    \end{equation*}
    
\end{enumerate}

\section{Implementación del método de Newton-Raphson}

Se implementa el método de Newton-Raphson, utilizando la siguiente función programada en \texttt{Python}.

\begin{minted}{python}
def newton_raphson(f, df, x0, tolerancia=1e-6, iteracion_max=1000):
    x = x0
    iteraciones = 0
    while abs(f(x)) > tolerancia and iteraciones < iteracion_max:
        h = f(x) / df(x)
        x = x - h
        iteraciones += 1
    if abs(f(x)) <= tolerancia:
        return x
    else:
        return None
\end{minted}

Finalmente, se obtienen las soluciones a cada ecuación:
\begin{enumerate}
    \item \textbf{Soluciones ecuación 1}: $x_0 = 1 , \, x_1 = 10$
    \item \textbf{Soluciones ecuación 2}: $x_0 \approx 0.64$
    \item \textbf{Soluciones ecuación 3}: $x_0 \approx 0.86 , \, x_1 \approx 3.42$
\end{enumerate}

\section{Conclusión}

En este capítulo, utilizamos el método de Newton-Raphson para encontrar numéricamente las soluciones de las ecuaciones planteadas en el problema.

\end{document}
