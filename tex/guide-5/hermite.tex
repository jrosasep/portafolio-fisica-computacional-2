\documentclass[../portafolio.tex]{subfiles}

\begin{document}

\chapter{Cálculo de los Polinomios de Hermite y sus Ceros}
\label{ch:hermite}
%%%%%%%%%%%%%%%%%%%%%%%%%%%%%%%%%%%%%%%%%%%%%%%%%%%%%%%%%%%%%%%%%%%%%%%%%%%%%%%%

\hfill \textbf{Fecha de la actividad:} 26 de noviembre de 2024

\medskip

En este capítulo:
\begin{itemize}
    \item Calculamos los primeros cinco polinomios de Hermite ($H_1$ a $H_5$) utilizando derivadas numéricas centradas.
    \item Encontramos los ceros de estos polinomios en el rango $-3 < x < 3$.
\end{itemize}

\section*{Polinomios de Hermite}

Los polinomios de Hermite son definidos por la relación de recurrencia:
\begin{equation*}
    H_{n+1}(x) = 2xH_n(x) - H'_n(x),
\end{equation*}
donde $H'_n(x)$ es la derivada del polinomio $H_n(x)$. Con la condición inicial:
\begin{equation*}
    H_0(x) = 1,
\end{equation*}
se puede construir la familia completa de polinomios.


\section{Derivada Numérica Centrada}
La derivada centrada para una función $f(x)$ en un punto $x_i$ se calcula como:
\begin{equation*}
    f'(x_i) \approx \frac{f(x_{i+1}) - f(x_{i-1})}{2h},
\end{equation*}
donde $h$ es el tamaño del paso entre los puntos de la discretización.

\section{Resultados}
Usando la relación de recurrencia y la derivada centrada, se obtuvieron los primeros cinco polinomios de Hermite. En la Figura \ref{fig:hermite-polinomios}, se grafican estos polinomios, y se marcan los ceros correspondientes en el rango $-3 < x < 3$.

\begin{figure}[H]
    \centering
 %   \includegraphics[width=0.8\textwidth]{hermite_polinomios.png}
    \caption{Primeros cinco polinomios de Hermite y sus ceros.}
    \label{fig:hermite-polinomios}
\end{figure}

En la Tabla \ref{tab:hermite-ceros}, se presentan los ceros encontrados para cada polinomio.

\begin{table}[H]
    \centering
    \begin{tabular}{|c|c|}
        \hline
        Polinomio & Ceros \\
        \hline
        $H_1(x)$ & $x = 0$ \\
        $H_2(x)$ & $x = \pm 0.707$ \\
        $H_3(x)$ & $x = 0, \pm 1.225$ \\
        $H_4(x)$ & $x = \pm 0.524, \pm 1.651$ \\
        $H_5(x)$ & $x = 0, \pm 0.959, \pm 2.020$ \\
        \hline
    \end{tabular}
    \caption{Ceros de los primeros cinco polinomios de Hermite.}
    \label{tab:hermite-ceros}
\end{table}

\section{Conclusiones}
Los primeros cinco polinomios de Hermite fueron calculados utilizando derivadas numéricas centradas. Los ceros fueron encontrados en el rango $-3 < x < 3$, y los resultados coinciden con los valores esperados teóricos para estos polinomios. Estos resultados son fundamentales para aplicaciones físicas, como la descripción de los estados del oscilador armónico cuántico.

\end{document}
