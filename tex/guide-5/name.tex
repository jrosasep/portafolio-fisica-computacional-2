\documentclass[../portafolio.tex]{subfiles}

\begin{document}

\chapter{Cálculo de los ceros de los polinomios de Legendre}
\label{ch:metodos-iterativos}
%%%%%%%%%%%%%%%%%%%%%%%%%%%%%%%%%%%%%%%%%%%%%%%%%%%%%%%%%%%%%%%%%%%%%%%%%%%%%%%%

\hfill \textbf{Fecha de la actividad:} 28 de noviembre de 2024

\medskip

\section{Definición de los polinomios de Legendre}
Los polinomios de Legendre \(P_n(x)\) son soluciones a la ecuación diferencial de Legendre, ampliamente utilizadas en problemas físicos. Su expresión general está dada por:
\begin{equation}
    P_n(x) = \frac{1}{2^n n!} \frac{d^n}{dx^n} \left( x^2 - 1 \right)^n.
\end{equation}

En este problema, trabajamos con los polinomios de grado \(n = 2, 3, 4\) y buscamos sus ceros en el rango \(|x| < 1\).

\section{Cálculo de los ceros}
Los ceros de los polinomios se determinaron usando la función de Python:
\texttt{numpy.polynomial.legendre.legroots}. Los resultados son los siguientes:

\begin{itemize}
    \item Para \(P_2(x)\):
    \begin{equation}
        \text{Ceros: } x = -0.5774, \, 0.5774.
    \end{equation}
    
    \item Para \(P_3(x)\):
    \begin{equation}
        \text{Ceros: } x = -0.7746, \, 0, \, 0.7746.
    \end{equation}

    \item Para \(P_4(x)\):
    \begin{equation}
        \text{Ceros: } x = -0.8611, \, -0.3399, \, 0.3399, \, 0.8611.
    \end{equation}
\end{itemize}

\section{Derivada numérica}
Se calculó la derivada de los polinomios mediante un esquema de diferencias centradas:
\begin{equation}
    f'(x) \approx \frac{f(x + h) - f(x - h)}{2h},
\end{equation}
donde \(h = 10^{-5}\). Esto permite obtener una aproximación eficiente de las derivadas.

\section{Conclusión}
Los resultados obtenidos coinciden con los valores esperados y validados mediante \texttt{numpy.polynomial.legendre.legroots}. Este análisis demuestra la utilidad de los polinomios de Legendre en aplicaciones físicas, y la derivada numérica valida las soluciones calculadas.

\end{document}