\documentclass[../portafolio.tex]{subfiles}

\begin{document}

\chapter{Demostración de una Aproximación Numérica de \textit{f'(x)} usando Series de Taylor}
\label{ch:taylor-diferenciacion}
%%%%%%%%%%%%%%%%%%%%%%%%%%%%%%%%%%%%%%%%%%%%%%%%%%%%%%%%%%%%%%%%%%%%%%%%%%%%%%%%

\hfill \textbf{Fecha de la actividad:} 4 de noviembre de 2024

\medskip

    En este capítulo, utilizaremos series de Taylor para derivar una aproximación numérica de segundo orden, $ O(h^2) $, para la primera derivada de una función $ f : \mathbb{R} \rightarrow \mathbb{R}$ que sea suficientemente suave alrededor de un punto $x$. Esta aproximación se expresará en términos de los valores $f(x)$, $ f(x + h) $ y $ f(x + 2h) $.
    
\section{Análisis}

    Conocemos una aproximación numérica de segundo orden para la primera derivada \cite{wiki:diferencia_finita} en términos de $f(x)$, $f(x + h)$ y $f(x + 2h)$:
\begin{equation}
    \label{eq:aproximación}
    f'(x) \approx \frac{4f(x + h) - f(x + 2h) - 3f(x)}{2h} \, \, .
\end{equation}

    Para demostrar la aproximación numérica de $f(x)$ en la ecuación (\ref{eq:aproximación}),  utilizamos las expansiones de Taylor de $f(x + h)$ y $f(x + 2h)$ alrededor de $x$. Sea $h\in \mathbb{R} : 0<h<1$.
\begin{equation}
    \label{eq:taylor0}
    f(x + h) = f(x) + h f'(x) + \frac{h^2}{2} f''(x) + \frac{h^3}{6} f'''(x) + \, \ldots
\end{equation}

\begin{equation}
    \label{eq:taylor1}
    f(x + 2h) = f(x) + 2h f'(x) + \frac{(2h)^2}{2} f''(x) + \frac{(2h)^3}{6} f'''(x) + \, \ldots
\end{equation}

    Multiplicamos la ecuación (\ref{eq:taylor0}) por $4$ y la ecuación (\ref{eq:taylor1}) por $-1$:
    
\begin{equation}
    \label{eq:taylor2}
    4 f(x + h) = 4 f(x) + 4 h f'(x) + 4 \frac{h^2}{2} f''(x) + 4 \frac{h^3}{6} f'''(x) + \, \ldots
\end{equation}

\begin{equation}
    \label{eq:taylor3}
    - f(x + 2h) = - f(x) - 2h f'(x) - \frac{(2h)^2}{2} f''(x) - \frac{(2h)^3}{6} f'''(x) - \, \ldots
\end{equation}

    Sumamos las ecuaciones (\ref{eq:taylor2}) y (\ref{eq:taylor3}) miembro a miembro y desarrollando, obtenemos:
\begin{equation*}
    4 f(x + h) - f(x + 2h) = 3 f(x) +  2 h f'(x) - \frac{4h^3}{6} f'''(x) + \, \ldots
\end{equation*}

    Despejamos $f'(x)$:
\begin{equation*}
    f'(x) = \frac{4f(x + h) - f(x + 2h) - 3f(x)}{2h} + \frac{2h^2}{6} f'''(x) + \, \ldots
\end{equation*}

    Sea $\xi \in \mathbb{R} : x-h < \xi < x+h $. Se tiene:
\begin{equation}
    \label{eq:aprox_end}
    f'(x) \approx  \frac{4f(x + h) - f(x + 2h) - 3f(x)}{2h} + \frac{2h^2}{6} f'''(\xi)
\end{equation}

    donde el término adicional $\frac{2h^2}{6} f'''(\xi)$ representa el error de truncamiento, de orden $O(h^2)$.

    Así, observamos que la expresión (\ref{eq:aprox_end}) es una aproximación de segundo orden para \( f'(x) \). Finalmente, omitiendo el error de truncamiento, obtenemos la diferencia aproximada:

\[
f'(x) \approx \frac{-3f(x) + 4f(x + h) - f(x + 2h)}{2h}.
\]

\section{Conclusión}

    En este capitulo, aprendí a demostrar una aproximación numérica de segundo orden para $f'(x)$ mediante series de Taylor, tomando los valores $f(x)$, $f(x+h)$ y $f(x+2h)$. Este proceso destacó la utilidad de las series de Taylor para mejorar la precisión en aproximaciones numéricas.
    
\end{document}
