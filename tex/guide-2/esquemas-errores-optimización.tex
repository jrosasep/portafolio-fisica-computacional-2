\documentclass[../portafolio.tex]{subfiles}

\begin{document}

\chapter{Cálculo Numérico de Derivadas: Esquemas, Errores y Optimización}
\label{ch:esquemas-errores-optimización}
%%%%%%%%%%%%%%%%%%%%%%%%%%%%%%%%%%%%%%%%%%%%%%%%%%%%%%%%%%%%%%%%%%%%%%%%%%%%%%%%

\hfill \textbf{Fecha de la actividad:} 20 de noviembre de 2024

\medskip

En este apartado se emplearán expansiones en series de Taylor de una función $f(x)$ para derivar tres esquemas de cálculo de la primera derivada numérica. Los esquemas considerados son los siguientes:

\begin{itemize}
    \item \textbf{Diferencia adelantada:} $f'(x) \approx \frac{f(x+h) - f(x)}{h}$.
    \item \textbf{Diferencia retrasada:} $f'(x) \approx \frac{f(x) - f(x-h)}{h}$.
    \item \textbf{Diferencia centrada:} $f'(x) \approx \frac{f(x+h) - f(x-h)}{2h}$.
\end{itemize}

Adicionalmente, se llevaron a cabo las siguientes tareas:
\begin{enumerate}
    \item Determinar el orden del error de truncamiento en cada uno de los esquemas anteriores.
    \item Analizar cuál de estos esquemas es más conveniente para estimar la derivada numérica.
    \item Considerando el error de redondeo de la máquina, calcular el valor óptimo de $h$ que minimiza el error en cada esquema y comparar los resultados obtenidos.
\end{enumerate}

Sea $h \in \mathbb{R} : 0 < h < 1$.

\section{Derivada adelantada}

Expandiendo $f(x+h)$ en series de Taylor:

\begin{equation}\label{eq:1-adelantada}
f(x+h) = f(x) + h f'(x) + \frac{h^2}{2!} f''(x) + \frac{h^3}{3!} f'''(x) + \cdots
\end{equation}

Restando en (\ref{eq:1-adelantada}) el polinomio $\left[ f(x) + \frac{h^2}{2!} f''(x) + \frac{h^3}{3!} f'''(x) + \cdots \right]$, se obtiene:

\begin{equation*}
f(x+h) - \left[ f(x) + \frac{h^2}{2!} f''(x) + \frac{h^3}{3!} f'''(x) + \cdots \right] = h f'(x)
\end{equation*}

Se reordenan términos:

\begin{equation*}
h f'(x) = f(x+h) - f(x) - \left[\frac{h^2}{2!} f''(x) + \frac{h^3}{3!} f'''(x) + \cdots \right]
\end{equation*}

Dividimos la ecuación por $h$ y despejamos $f'(x)$, obtenemos:

\begin{equation}\label{eq:2-adelantada}
f'(x) = \frac{f(x+h) - f(x)}{h} - \left[\frac{h}{2!} f''(x) + \frac{h^2}{3!} f'''(x) + \cdots \right]
\end{equation}

Así, se obtiene una forma para la \textbf{derivada adelantada} de $f(x)$ en términos de $f(x)$ y $f(x+h)$.

\section{Derivada retrasada}

Expandiendo $f(x-h)$ en series de Taylor:

\begin{equation}\label{eq:1-retrasada}
f(x-h) = f(x) - h f'(x) + \frac{h^2}{2!} f''(x) - \frac{h^3}{3!} f'''(x) + \cdots
\end{equation}

Restando en (\ref{eq:1-retrasada}) el polinomio $\left[ f(x) + \frac{h^2}{2!} f''(x) - \frac{h^3}{3!} f'''(x) + \cdots \right]$, se obtiene:

\begin{equation*}
f(x-h) - \left[ f(x) - \frac{h^2}{2!} f''(x) - \frac{h^3}{3!} f'''(x) + \cdots \right] = - h f'(x)
\end{equation*}

Multiplicando la ecuación por $-1$ y reordenando términos:

\begin{equation*}
h f'(x) = f(x) - f(x-h) + \left[\frac{h^2}{2!} f''(x) - \frac{h^3}{3!} f'''(x) + \cdots \right]
\end{equation*}

Dividiendo la ecuación por $h$ y despejamos $f'(x)$, obtenemos:

\begin{equation}\label{eq:2-retrasada}
f'(x) = \frac{f(x) - f(x-h)}{h} + \left[\frac{h}{2!} f''(x) - \frac{h^2}{3!} f'''(x) + \cdots \right]
\end{equation}

Así, obtenemos una forma para la \textbf{derivada retrasada} de $f(x)$ en términos de $f(x)$ y $f(x-h)$.

\section{Derivada centrada}

Consideremos las expansiones en series de Taylor de $f(x+h)$ y $f(x-h)$, obtenidas en las ecuaciones (\ref{eq:1-adelantada}) y (\ref{eq:1-retrasada}), respectivamente.

Restando las ecuaciones (\ref{eq:1-adelantada}) y (\ref{eq:1-retrasada}) en ese orden, obtenemos:

\begin{equation*}
f(x+h) - f(x-h) = 2h f'(x) + 2\frac{h^3}{3!} f'''(x) + 2\frac{h^5}{5!}f^{(5)}(x) + \cdots
\end{equation*}

Dividiendo la ecuación por $2h$ y despejamos $f'(x)$, reordenamos terminos y obtenemos:

\begin{equation}\label{eq:centrada}
f'(x) = \frac{f(x+h) - f(x-h)}{2h} - \left[ \frac{h^2}{3!} f'''(x) + \frac{h^4}{5!}f^{(5)}(x) + \cdots \right]
\end{equation}

Así, obtenemos una forma para la \textbf{derivada centrada} de $f(x)$ en términos de $f(x+h)$ y $f(x-h)$.

\section{Orden del error de truncamiento en los esquemas numéricos}

Para el esquema de la \textbf{derivada adelantada} de $f(x)$. Notamos en la ecuación (\ref{eq:2-adelantada}) que el error de truncamiento será del orden $O(h)$. Luego:

\begin{equation}
f'(x) \approx \frac{f(x+h) - f(x)}{h} + \frac{h}{2!} f''(\xi)
\end{equation}

Para el esquema de la \textbf{derivada retrasada} de $f(x)$. Notamos en la ecuación (\ref{eq:2-retrasada}) que el error de truncamiento, nuevamente, será del orden $O(h)$. Luego:

\begin{equation}
f'(x) \approx \frac{f(x) - f(x-h)}{h} + \frac{h}{2!} f''(\xi)
\end{equation}

Para el esquema de la \textbf{derivada centrada} de $f(x)$. Notamos en la ecuación (\ref{eq:centrada}) que el error de truncamiento esta vez será del orden $O(h^2)$. Luego:

\begin{equation}
f'(x) \approx \frac{f(x+h) - f(x-h)}{2h} - \frac{h^2}{3!} f'''(\xi)
\end{equation}

\subsection{¿Cuál de estos esquemas es más conveniente para estimar la derivada numérica?}

Como el esquema de la derivada centrada de $f(x)$ presenta un error de truncamiento $O(h^2)$, con $0<h<1$, es más conveniente el uso de este respecto a los otros para estimar la derivada numérica. Dado que el error en el esquema centrado disminuye más rápidamente a medida que $h$ se reduce, resulta más preciso para valores pequeños de $h$.

\section{Cálculo de \textit{h} óptimo para el esquema de la derivada retrasada}

Consideramos $\hat{a} = a(1 + \varepsilon)$, donde $\hat{a}$ representa el valor de $a$ registrado por el computador, afectado por un error de redondeo $\varepsilon$, el cual puede depender de $a$. Para el esquema de la derivada retrasada, usamos la aproximación:

\begin{equation} \label{eq-optimo:retrasada}
f'(x) \approx \frac{f(x) - f(x-h)}{h} + \frac{h}{2!}f''(\xi),
\end{equation}

donde los valores de \(f(x-h)\) y \(f(x)\) se redondean como:

\begin{equation} \label{eq-optimo:computador}
\hat{f}(x-h) = f(x-h)(1 + \varepsilon_2), \quad \text{y} \quad \hat{f}(x) = f(x)(1 + \varepsilon_1).
\end{equation}

Sustituyendo estos valores en (\ref{eq-optimo:retrasada}), obtenemos:
\begin{equation}
f'(x) - \frac{\hat{f}(x) - \hat{f}(x-h)}{h} = \frac{h}{2!} f''(\xi)
\end{equation}

Expandiendo y simplificando:

\begin{equation*}
f'(x) - \frac{f(x)(1 + \varepsilon_1) - f(x-h)(1 + \varepsilon_2)}{h} = \frac{h}{2!} f''(\xi)
\end{equation*}

\begin{equation*}
f'(x) - \frac{f(x) - f(x-h)}{h} = \frac{h}{2!} f''(\xi) + \frac{f(x)\varepsilon_1 - f(x-h)\varepsilon_2}{h}
\end{equation*}

Aplicando la desigualdad triangular:

\begin{equation*}
\left|f'(x) - \frac{f(x) - f(x-h)}{h}\right| \leq \frac{h}{2} \left|f''(\xi)\right| + \frac{|\varepsilon_1|}{h} \left|f(x)\right| + \frac{|\varepsilon_2|}{h} \left|f(x-h)\right|
\end{equation*}

Podemos asumir que:

\begin{enumerate}
    \item Existe un valor $\varepsilon^*$ tal que $\varepsilon^* \gg  \left|\varepsilon_1 \right|$ y $\varepsilon^* \gg \left| \varepsilon_2\right|$.
    \item Si $h \ll 1$, entonces $\xi \approx x \quad \wedge \quad \left|f(x-h)\right| \approx \left|f(x)\right|$.
\end{enumerate}

Entonces:

\begin{equation*}
\left|f'(x) - \frac{f(x) - f(x-h)}{h}\right| \leq \frac{h}{2} \left|f''(x)\right| + \frac{2 \varepsilon^*}{h}\left|f(x)\right|
\end{equation*}

De aquí, el \textbf{error absoluto} (analítico) es:

\begin{equation}    \label{eq-optimo:absoluto}
E_{\text{abs}} \leq \frac{2\varepsilon^*}{h} \left|f(x)\right| + \frac{h}{2} \left|f''(x)\right|
\end{equation}

El error absoluto \textbf{derivado} es:

\begin{equation} \label{eq-optimo:absoluto-prima}
E'_{\text{abs}} \leq -\frac{2\varepsilon^*}{h^2} \left|f(x)\right| + \frac{1}{2} \left|f''(x)\right|
\end{equation}

Por otro lado, el \textbf{valor óptimo} de $h$ se obtiene a partir de $E'_{\text{abs}} = 0$.

\begin{equation*}
-\frac{2\varepsilon^*}{h^2} \left|f(x)\right| + \frac{1}{2} \left|f''(x)\right| = 0
\end{equation*}

Despejamos $h$:

\begin{equation}
h = \sqrt{\frac{4\varepsilon^* \left|f(x)\right|}{\left|f''(x)\right|}}
\end{equation}
Se obtiene el valor optimo de $h$ para el esquema de la  derivada retrasada.

\section{\textit{h} óptimo para el esquema de la derivada adelantada}

Ya sabemos que el esquema de la derivada adelantada es:

\begin{equation} \label{eq-optimo:adelantada}
f'(x) \approx \frac{f(x+h) - f(x)}{h} + \frac{h}{2} f''(\xi)
\end{equation}

Donde el termino $\frac{h}{2} f''(\xi)$ es el error de truncamiento que depende de la segunda derivada de  $f(x)$. 
Para considerar los errores de redondeo de la máquina, utilizamos las siguientes expresiones:

\begin{equation} \label{eq-optimo:computador-adelantada}
\hat{f}(x+h) = f(x+h)(1 + \varepsilon_2), \quad \text{y} \quad \hat{f}(x) = f(x)(1 + \varepsilon_1).
\end{equation}

Consideramos los errores de redondeo de la maquina (\ref{eq-optimo:computador-adelantada}) en el esquema de la derivada adelantada (\ref{eq-optimo:adelantada}), y expandimos términos:

\begin{align*}
    f'(x) & = \frac{\hat{f}(x+h) - \hat{f}(x) }{h} + \frac{h}{2}f''(\xi) \\
     & = \frac{f(x+h)(1 + \varepsilon_2) - f(x)(1 + \varepsilon_1) }{h} + \frac{h}{2}f''(\xi) \\
     & = \frac{f(x+h) - f(x)}{h} + \frac{ \varepsilon_2f(x+h) - \varepsilon_1f(x) }{h} + \frac{h}{2}f''(\xi)
\end{align*}

Restamos $\frac{f(x+h) - f(x)}{h}$ en ambos miembros de la ecuación, obtenemos:

\begin{equation*}
     f'(x) - \frac{f(x+h) - f(x)}{h}  = \frac{ \varepsilon_2f(x+h) - \varepsilon_1f(x) }{h} + \frac{h}{2}f''(\xi)
\end{equation*}

Aplicando desigualdad triangular:
\begin{equation*}
         \left| f'(x) - \frac{f(x+h) - f(x)}{h} \right| \leq \left| \frac{ \varepsilon_2f(x+h) - \varepsilon_1f(x) }{h} \right| + \left| \frac{h}{2}f''(\xi) \right|
\end{equation*}

Podemos asumir que:

\begin{enumerate}
    \item Existe un valor $\varepsilon^*$ tal que $\varepsilon^* \gg  \left|\varepsilon_1 \right|$ y $\varepsilon^* \gg \left| \varepsilon_2\right|$.
    \item Si $h \ll 1$, entonces $\xi \approx x \quad \wedge \quad \left|f(x+h)\right| \approx \left|f(x)\right|$.
\end{enumerate}

Luego, tenemos para el \textbf{error absoluto}:

\begin{align*}     
     E_{\text{abs}} & \leq \frac{ \varepsilon_2}{h} \left| f(x+h) \right| + \frac{\varepsilon_1}{h} \left|f(x) \right|  + \frac{h}{2} \left|  f''(\xi) \right| \\
     & \leq \frac{ \varepsilon^*}{h} \left| f(x) \right| + \frac{\varepsilon^*}{h} \left|f(x) \right|  + \frac{h}{2} \left|  f''(\xi) \right| \\
     & = \frac{ 2\varepsilon^*}{h} \left| f(x) \right| + \frac{h}{2} \left|  f''(\xi) \right| \\
\end{align*}

Así, deducimos el \textbf{error absoluto} del esquema de derivada adelantada.

\begin{equation*}
     E_{\text{abs}} \leq \frac{ 2\varepsilon^*}{h} \left| f(x) \right| + \frac{h}{2} \left|  f''(\xi) \right|
\end{equation*} 

Derivamos respecto a $h$ el error absoluto $E_{\text{abs}}$.

\begin{align*}     
     E'_{\text{abs}} & \leq -\frac{ \varepsilon^*}{h^2} \left| f(x) \right| + \frac{1}{2} \left|  f''(\xi) \right| \\
\end{align*}

Para encontrar el valor optimo de $h$, resolvemos la ecuación $E'_{\text{abs}} = 0$ para optimizar el error, obtenemos la siguiente ecuación:

\begin{align*}
     0 & = - \frac{ \varepsilon^*}{h^2} \left| f(x) \right| + \frac{1}{2} \left|  f''(\xi) \right|
\end{align*}

Despejando \(h\), encontramos el valor óptimo:

\begin{equation} \label{eq-optimo:h_adelantada}
    h = \sqrt{\frac{4\varepsilon^* \left|f(x)\right|}{\left|f''(x)\right|}}
\end{equation}


%%%%%%%%%%%%%%%%%%%%%%%%%%%%%%%%%%%%%%%%%%%%%%%%%%%%%%%%%%%%%%%%%%%%%%%%%

\section{\textit{h} óptimo para el esquema de la derivada centrada}

Conocemos el esquema de la derivada centrada:

\begin{equation}\label{eq-optimo:centrada}
f'(x) \approx \frac{f(x+h) - f(x-h)}{2h} - \frac{h^2}{3!} f'''(\xi)
\end{equation}

Donde el termino $\frac{h^2}{3!} f'''(\xi)$ es el error de truncamiento que depende de la tercera derivada de  $f(x)$. 
Para considerar los errores de redondeo de la máquina, utilizamos las siguientes expresiones:

\begin{equation} \label{eq-optimo:computador-centrada}
\hat{f}(x+h) = f(x+h)(1 + \varepsilon_2), \quad \text{y} \quad \hat{f}(x-h) = f(x-h)(1 + \varepsilon_1).
\end{equation}

Consideramos los errores de redondeo de la maquina (\ref{eq-optimo:computador-centrada}) en el esquema de la derivada centrada (\ref{eq-optimo:centrada}), y expandimos términos:

\begin{align*}
    f'(x) & = \frac{\hat{f}(x+h) - \hat{f}(x-h) }{2h} + \frac{h^2}{3!} f'''(\xi) \\
     & = \frac{f(x+h)(1 + \varepsilon_2) - f(x-h)(1 + \varepsilon_1) }{2h} + \frac{h^2}{3!} f'''(\xi) \\
     & = \frac{f(x+h) - f(x)}{2h} + \frac{ \varepsilon_2f(x+h) - \varepsilon_1f(x) }{2h} + \frac{h^2}{3!} f'''(\xi)
\end{align*}

Restamos $\frac{f(x+h) - f(x-h)}{2h}$ en ambos miembros de la ecuación, obtenemos:

\begin{equation*}
     f'(x) - \frac{f(x+h) - f(x-h)}{2h}  = \frac{ \varepsilon_2f(x+h) - \varepsilon_1f(x-h) }{2h} +\frac{h^2}{3!} f'''(\xi)
\end{equation*}

Aplicando desigualdad triangular, y asumiendo que:
\begin{enumerate}
    \item Existe un valor $\varepsilon^*$ tal que $\varepsilon^* \gg  \left|\varepsilon_1 \right|$ y $\varepsilon^* \gg \left| \varepsilon_2\right|$.
    \item Si $h \ll 1$, entonces $\xi \approx x \quad \wedge \quad \left|f(x+h)\right| \approx \left|f(x)\right|\quad \wedge \quad \left|f(x-h)\right| \approx \left|f(x)\right|$.
\end{enumerate}
Se tiene el siguiente desarrollo:
\begin{align*}
         \left| f'(x) - \frac{f(x+h) - f(x-h)}{2h} \right| & \leq \left| \frac{ \varepsilon_2f(x+h) - \varepsilon_1f(x-h) }{2h} \right| + \left| \frac{h^2}{3!} f'''(\xi) \right| \\
         & \leq \left| \frac{ \varepsilon_2f(x+h)}{2h} \right| +  \left| \frac{\varepsilon_1f(x-h) }{2h} \right| + \left| \frac{h^2}{6} f'''(\xi) \right| \\
         & =  \frac{ \varepsilon_2}{2h} \left| f(x+h) \right| +  \frac{\varepsilon_1}{2h}  \left|f(x-h)\right|  + \frac{h^2}{6} \left| f'''(\xi) \right| \\
              & \leq  \frac{ \varepsilon^*}{2h} \left| f(x) \right| +  \frac{\varepsilon^*}{2h}  \left|f(x)\right|  + \frac{h^2}{6} \left| f'''(\xi) \right| \\
         & = \frac{\varepsilon^*}{h} \left| f(x) \right| + \frac{h^2}{6} \left| f'''(\xi) \right|
\end{align*}

Notamos que el \textbf{error absoluto} estara dado por:
\begin{equation}
E_{\text{abs}} \leq \frac{h^2}{6} \lvert f'''(\xi) \rvert + \frac{\varepsilon^*}{h} \lvert f(x) \rvert.
\end{equation}

Para el valor optimo de $h$, nuevamente derivamos el error absoluto respecto a $h$ y resolvemos $E'_{\text{abs}} = 0$:
\begin{equation}
-\frac{ \varepsilon^*}{h^2} \lvert f(x) \rvert + \frac{h}{3} \lvert f'''(x) \rvert = 0.
\end{equation}

Resolviendo para $h$, obtenemos:
\begin{equation}
h = \left( \frac{3\varepsilon^* \lvert f(x) \rvert}{\lvert f'''(x) \rvert} \right)^{1/3}.
\end{equation}

\section{Comparación de resultados}

Se observa que el valor óptimo de $h$ es el mismo para los esquemas de derivada adelantada y retrasada, pero difiere del obtenido para el esquema de derivada centrada. En todos los casos, el valor óptimo de $h$ se expresa como una raíz: cuadrada para los esquemas adelantado y retrasado, y cúbica para el esquema centrado. El argumento de la raíz incluye un factor proporcional a $\varepsilon^*$ y la relación entre los valores absolutos de $f(x)$ y sus derivadas: $f''(x)$ en el caso de la raíz cuadrada y $f'''(x)$ en el caso de la raíz cúbica.

\section{Conclusión}

El desarrollo de la actividad permite aprender a calcular derivadas numéricas mediante tres esquemas (adelantado, retrasado y centrado), analizar sus errores de truncamiento y redondeo, y determinar el valor óptimo de $h$ para minimizar el error total. Además, fomenta el uso de técnicas matemáticas como series de Taylor y análisis de la concavidad para evaluar y optimizar estos métodos, destacando la precisión superior del esquema centrado.

\end{document}