\documentclass[../portafolio.tex]{subfiles}

\begin{document}

\chapter*{Conclusiones}
\addcontentsline{toc}{chapter}{Conclusiones}
\markboth{Conclusiones}{Conclusiones}

\hfill \textbf{Fecha de presentación:} Viernes 29 de noviembre de 2024

\medskip

\chapter{Conclusión del portafolio}

%--------------------------------------------------------------------------------
\section{Resumen de los objetivos del portafolio}

El portafolio tuvo como objetivo principal documentar y reflejar el aprendizaje adquirido a lo largo del curso mediante la recopilación de evidencias significativas. Este instrumento permitió analizar de forma crítica los conceptos aprendidos, identificar fortalezas y áreas de mejora, así como reflexionar sobre la aplicación futura de los conocimientos adquiridos.

%--------------------------------------------------------------------------------
\section{Resumen de los contenidos}

En este portafolio se incluyen diversas evidencias de aprendizaje, cada una asociada a actividades realizadas durante el curso. Se implemento python para resolver numéricamente problemas matemáticos y físicos complejos. Se llevaron a cabo demostraciones matemáticas cuando fue solicitado.


%--------------------------------------------------------------------------------
\section{Autoevaluación del alumno/a}

Durante el curso, considero que mi desempeño fue mediocre. A futuro, me gustaría enfocarme más en distribuir mejor mi tiempo y profundizar en la comprensión teórica antes de abordar la implementación práctica. Los contenidos de este portafolio son aplicables a futuros proyectos, especialmente aquellos que involucren simulaciones numéricas y análisis de datos.

%--------------------------------------------------------------------------------
\section{Evaluación del curso}

Al inicio del curso, no esperaba aprender a resolver problemas numéricos y modelar sistemas dinámicos. Las actividades del portafolio me permitieron introducirme a técnicas avanzadas y herramientas de programación que desconocía.

Aspectos positivos del curso incluyen la integración de conceptos teóricos y prácticos, así como el enfoque en la validación de resultados. Sin embargo, considero que podría mejorarse la claridad en algunos materiales de referencia y en las instrucciones de las actividades.

En cuanto a las evidencias incluidas, creo que la simulación de la trayectoria del cometa fue la más relevante, ya que integró conocimientos de programación, física y matemáticas de forma aplicada. Esto la diferencia de otras actividades que fueron más específicas y conceptuales.

Finalmente, considero que el portafolio es una herramienta útil no solo para evaluar el aprendizaje, sino también para reflexionar y consolidar los conocimientos adquiridos.

%--------------------------------------------------------------------------------
\section{Sugerencias para futuras versiones del curso}

\begin{itemize}
    \item Incluir más ejemplos prácticos en los materiales de referencia.
    \item Proveer guías detalladas para la interpretación de resultados numéricos.
    \item Fomentar el trabajo colaborativo en las actividades del portafolio.
\end{itemize}


\end{document}