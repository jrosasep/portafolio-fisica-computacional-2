\documentclass[../portafolio.tex]{subfiles}

% Solo agregue paquetes en el preámbulo de ../portafolio.tex

\begin{document}

\chapter*{Información personal y académica}
\addcontentsline{toc}{chapter}{Información personal y académica}
\markboth{Información personal y académica}{Información personal y académica}


%%%%%%%%%%%%%%%%%%%%%%%%%%%%%%%%%%%%%%%%%%%%%%%%%%%%%%%%%%%%%%%%%%%%%%
% Llene todos los campos, respetando tildes, mayúsculas y minúsculas.
\section*{Datos personales}

\begin{description}
\item[{Nombre completo}] José Ignacio Rosas Sepúlveda % nombres y apellidos completos.
\item[{Matrícula}] 2023402508               % matrícula udec
\item[{Fecha de Nacimiento}] 2 de septiembre de 2000 % día de mes de año
\item[{Nacionalidad}] Chileno
\item[{E-Mail institucional}] \href{mailto:loki@asgard.mcu}{jrosas2022@udec.cl}
\end{description}


%%%%%%%%%%%%%%%%%%%%%%%%%%%%%%%%%%%%%%%%%%%%%%%%%%%%%%%%%%%%%%%%%%%%%%
\section*{Breve biografía académica}

Soy José Rosas, estudiante de segundo año de la carrera Ciencias Físicas, ingresado en el año 2023. Realicé mi educación media en el colegio Gran Bretaña de Concepción. Desde los 14 años me dediqué a la composición musical. En pandemia desarrolle un profundo interés por las ciencias. Mis objetivos académicos luego del pregrado son desarrollar mis conocimientos en física de partículas, me interesa investigar la fisica implicada en procesos bioquímicos. Por otra parte, deseo ampliar mis conocimientos en Biología para complementar mi formación científica.

%%%%%%%%%%%%%%%%%%%%%%%%%%%%%%%%%%%%%%%%%%%%%%%%%%%%%%%%%%%%%%%%%%%%%%
\section*{Visión general e interés sobre la asignatura}
Mi percepción inicial de Física Computacional II es que será clave para aplicar herramientas computacionales en problemas físicos complejos. La asignatura integra teoría, simulaciones y cálculos numéricos, fortaleciendo mi formación académica. Espero adquirir destrezas en software y algoritmos aplicados a la física, esenciales para abordar problemas multidisciplinarios. Confío en que este conocimiento será valioso en cursos avanzados y en mi futura carrera, especialmente en simulación y modelado físico. Me interesa particularmente el uso de técnicas numéricas en sistemas dinámicos.


%%%%%%%%%%%%%%%%%%%%%%%%%%%%%%%%%%%%%%%%%%%%%%%%%%%%%%%%%%%%%%%%%%%%%%
\section*{Resultados esperados de este portafolio}
Al completar este portafolio, espero consolidar mis habilidades en herramientas computacionales aplicadas a la física y en la resolución de problemas numéricos. Este documento organizará los conceptos aprendidos, facilitando su análisis y aplicación en diversos contextos. También será una referencia valiosa para mi formación futura, documentando ejemplos y metodologías reutilizables en proyectos académicos o profesionales. La autoevaluación y recopilación de evidencias me permitirán identificar áreas de mejora, evaluar mi progreso y fortalecer mi comprensión de los temas tratados.
\end{document}