\documentclass[../portafolio.tex]{subfiles}

\begin{document}

\chapter{Potencial electrostático de dos cargas puntuales sobre el Plano}
\label{ch:script-potencial-electrostatico}
%%%%%%%%%%%%%%%%%%%%%%%%%%%%%%%%%%%%%%%%%%%%%%%%%%%%%%%%%%%%%%%%%%%%%%%%%%%%%%%%

\begin{minted}{python}

import numpy as np
import matplotlib.pyplot as plt

a = 2    
q1 = 1e-9   
q2 = 1e-9   
k = 8.99e9  

pos_q1 = (-a/2, 0)  
pos_q2 = (a/2, 0)  

# Función para calcular el potencial electrostático en un punto (x, y)
def potencial(x, y, q, pos):
    rx, ry = x - pos[0], y - pos[1]
    r = np.sqrt(rx**2 + ry**2)
    return k * q / r

# Generamos el subconjunto del plano (x,y)
x = np.linspace(-5, 5, 400)     
y = np.linspace(-5, 5, 400)     

X, Y = np.meshgrid(x, y)   

# Calculamos el potencial total para los tres casos:
def calcular_potencial_total(q1, q2):
    V1 = potencial(X, Y, q1, pos_q1)    
    V2 = potencial(X, Y, q2, pos_q2)    
    return V1 + V2  

# Creamos una figura con subplots para los tres casos
fig, axes = plt.subplots(1, 3, figsize=(18, 6))

# Caso 1: Ambas cargas positivas
V = calcular_potencial_total(q1, q2)
axes[0].contourf(X, Y, V, levels=50, cmap="coolwarm")
axes[0].set_title("Ambas cargas positivas")
axes[0].set_xlabel("x (m)")
axes[0].set_ylabel("y (m)")

# Caso 2: Ambas cargas negativas
V = calcular_potencial_total(-q1, -q2)
axes[1].contourf(X, Y, V, levels=50, cmap="coolwarm")
axes[1].set_title("Ambas cargas negativas")
axes[1].set_xlabel("x (m)")

# Caso 3: Cargas opuestas
V = calcular_potencial_total(q1, -q2)
axes[2].contourf(X, Y, V, levels=50, cmap="coolwarm")
axes[2].set_title("Cargas opuestas")
axes[2].set_xlabel("x (m)")

# Ajuste de la visualización
for ax in axes:
    ax.set_aspect("equal")

plt.tight_layout()
plt.show()

    
\end{minted}

\end{document}