
\documentclass[../portafolio.tex]{subfiles}

\begin{document}

\chapter{Cálculo y análisis de los números de Catalán}
\label{ch:numeros-catalan}

%%%%%%%%%%%%%%%%%%%%%%%%%%%%%%%%%%%%%%%%%%%%%%%%%%%%%%%%%%%%%%%%%%%%%%%%%%%%%%%%

\hfill \textbf{Fecha de la actividad:} 16 de noviembre de 2024

\medskip

Este capítulo aborda las propiedades fundamentales de los números de Catalán, definidos como: 

\begin{equation}\label{eq:catalan}
C_n = \frac{(2n)!}{(n+1)!n!}, \quad n \geq 0.
\end{equation}

Se demuestra que esta definición puede expresarse recursivamente como: 

\begin{equation}
C_0 = 1, \quad C_{n+1} = \frac{4n+2}{n+2} C_n.
\end{equation}

Además, se utiliza \texttt{Python} para:  
\begin{itemize}
    \item Graficar todos los números de Catalán que satisfacen $C_n < M$, con $M > 10^{15}$, empleando precisión simple (32 bits).
    \item Comparar los resultados obtenidos mediante la definición exacta y la relación recursiva.
    \item Verificar gráficamente el comportamiento asintótico de la serie para $n \gg 1$.
\end{itemize}

Se incluye un análisis visual que destaca la relación entre los valores exactos y su aproximación, resaltando las propiedades de esta importante secuencia combinatoria.


\section{Definición y forma de recurrencia}

Se puede demostrar que la definición (\ref{eq:catalan}) es equivalente a la recurrencia:

\begin{equation}
C_0 = 1, \quad C_{n+1} = \frac{4n + 2}{n + 2} C_n
\end{equation}

\subsection*{Demostración}

\begin{enumerate}
    \item Observemos que, según la definición (\ref{eq:catalan}), para $n = 0$ se tiene:
    
    \begin{align*}
        C_0 & = \frac{(2 \cdot 0)!}{(0 + 1)! \cdot 0!} \\
            & = \frac{(0)!}{1! \cdot 0!} \\
            & = \frac{1}{1 \cdot 1} \\
            & = 1.
    \end{align*}
    
    Por lo tanto, queda demostrado que $C_0 = 1$.
\item La recurrencia $C_{n+1} = \frac{4n + 2}{n + 2} C_n$ se demuestra mediante el principio de inducción matemática. Procedemos de la siguiente manera:

\begin{itemize}
    \item \textbf{Hipótesis inductiva:} Suponemos que la fórmula es válida para un entero no negativo $k$, es decir, que $C_k = \frac{(2k)!}{(k+1)! \, k!}$.
    \item \textbf{Tesis inductiva:} Queremos demostrar que $C_{k+1} = \frac{4k + 2}{k + 2} C_k$.
\end{itemize}

A partir de la definición de los números de Catalán, evaluamos $C_{k+1}$:

\begin{align}
    C_{k+1} & = \frac{(2(k+1))!}{((k+1)+1)! \, (k+1)!} \\
    & = \frac{(2k+2)!}{(k+2)! \, (k+1)!},
\end{align}
donde simplemente hemos expandido $2(k+1)$ como $2k+2$ y $(k+1)+1$ como $k+2$ en los factoriales. A continuación, separamos el factorial $(2k+2)!$ utilizando la propiedad de los factoriales $(n+1)! = (n+1) \cdot n!$:

\begin{align}
    C_{k+1} & = \frac{(2k)! \, (2k+1) \, (2k+2)}{(k+1)! \, (k+2)!} \\
    & = \frac{(2k)! \, (2k+1) \, (2k+2)}{(k+1)! \, (k+2) \, k! \, (k+1)},
\end{align}
donde el término $(k+2)!$ fue descompuesto como $(k+2) \cdot (k+1)!$. Luego, reorganizamos los términos agrupando los factores comunes:

\begin{align}
    C_{k+1} & = \frac{(2k)! \, (2k+1) \, 2 \, (k+1)}{(k+1)! \, (k+2) \, k! \, (k+1)}.
\end{align}
Aquí, se utilizó $2k+2 = 2 \cdot (k+1)$ para simplificar el último factor. Al cancelar $(k+1)$ en el numerador y denominador, el resultado queda como:

\begin{align}
    C_{k+1} & = \frac{(2k)! \, (4k+2)}{(k+1)! \, (k+2) \, k!}.
\end{align}
Luego, identificamos que el primer término $\frac{(2k)!}{(k+1)! \, k!}$ corresponde a $C_k$, lo que permite expresar la ecuación como:

\begin{align}
    C_{k+1} & = \frac{(2k)!}{(k+1)! \, k!} \cdot \frac{4k+2}{k+2} \\
    & = C_k \cdot \frac{4k+2}{k+2}.
\end{align}

Por lo tanto, hemos demostrado que:

\begin{equation}\label{eq:catalan_recursividad}
    C_{n+1} = \frac{4n+2}{n+2} C_n ,\quad  \forall n \in \mathbb{N}_0.
\end{equation}

\end{enumerate}

De esta forma, se valida que los números de Catalán pueden expresarse de manera equivalente mediante la relación de recurrencia. Cabe notar que, al evaluar $n = 0$, se obtiene directamente que $C_0 = 1$, lo cual cierra la demostración de la base inductiva.

%%%%%%%%%%%%%%%%%%%%%%%%%%%%%%%%%%%%%%%%%%%%%%%%%%%%%%%%%%%%%%%%%%%%%%%%%%%%%%%%%%%%%%%%%%%%%%%%%%%%%%%%%%%%%%%%%%%%%%%%%%%%%%%%%%%%%%%%%%%%

\section{Gráfica y comparación}

Para calcular los números de Catalán y graficarlos utilizaremos \texttt{Python}, empleando las bibliotecas \texttt{numpy} y \texttt{matplotlib.pyplot}, para calcular hasta $C_n < M$, con una cota $M$ dada por el usuario. 

Primero, definimos una función para calcular el operador factorial.
   
\begin{minted}{python}
def fact(n):
    fact=1
    for n in list(range(0,int(n+1),1)):
        if n==0 or n==1:
            fact*=1
        elif n>1:
            fact*= n
    return fact
\end{minted}

Definimos un par de funciones para calcular el n-esimo numero de Catalán, asegurándonos de usar precisión \textbf{simple} con la función \texttt{float32} de \texttt{numpy}, por:
\begin{itemize}
    \item \textbf{Definición} (\ref{eq:catalan}):
        \begin{minted}{python}
def catalan_def(n):
    catalan_n = np.float32 ( fact(2*n)/( fact(n+1)* fact(n) ) )
    return catalan_n
        \end{minted}    
    \item \textbf{Recursividad} (\ref{eq:catalan_recursividad}):
    
    \begin{minted}{python}
def catalan_rec(n):
    catalan_0 = 1
    for n in list( range( 0, n, 1) ):
        catalan_rec = np.float32( catalan_0 * ( 4*n + 2 ) / ( n + 2 ) )
        catalan_0 = catalan_rec
    return catalan_0
    \end{minted}
    
\end{itemize}
Empleando la función \texttt{input} se programa \texttt{M} para que sea una cota ingresada por el usuario. La función \texttt{float} interpreta el valor ingresado por el usuario como un float, si y solo si la sintaxis es correcta.
\begin{minted}{python} 
M = float(input("Ingrese el valor de M: "))
\end{minted}

Establecemos las siguientes listas vacías para almacenar los resultados:
\begin{itemize}
    \item \texttt{n\_values}: contendrá los números naturales 
    \item \texttt{list\_catalan\_def}: almacenará los valores calculados por definición.
    \item \texttt{list\_catalan\_rec}: guardará los valores obtenidos mediante recurrencia.
\end{itemize}
El código es el siguiente:
\begin{minted}{python}
n_values = []
list_catalan_def = []
list_catalan_rec = []
\end{minted}

Se define una variable \texttt{n = 0} para iniciar un ciclo \texttt{while}, donde se calculan los números de Catalán $C_n $ mediante la definición (\ref{eq:catalan}) y la recurrencia (\ref{eq:catalan_recursividad}). En cada iteración, los valores calculados y el índice $n$ se almacenan en listas respectivas. El ciclo finaliza cuando $ M \geq C_n $.
  
\begin{minted}{python}
# Cálculo iterativo de los números de Catalán
n = 0
while True:
    C_def = catalan_def(n)  # Número de Catalán por definición
    C_rec = catalan_rec(n)  # Número de Catalán por recurrencia

    # Condición para detener el ciclo si se supera la cota M
    if C_def >= M or C_rec >= M:
        break

    # Almacenamos los valores calculados y el índice actual
    list_catalan_def.append(C_def)
    list_catalan_rec.append(C_rec)
    n_values.append(n)
    
    # Incrementamos el índice
    n += 1
\end{minted}
Para visualizar los números de Catalán que cumplen $M \geq C_n$, se generó un gráfico empleando \texttt{matplotlib.pyplot}. En este se comparan los resultados obtenidos mediante las expresiones (\ref{eq:catalan}) y (\ref{eq:catalan_recursividad}). Además, se incluye una línea horizontal roja que representa la cota $M$. El eje $x$ corresponde a los primeros $n \in \mathbb{N}_0$, mientras que el eje $y$ muestra los valores de $C_n$ calculados bajo dicha restricción.

\begin{minted}{python}

# Gráfico de los números de Catalán calculados por ambos métodos
plt.plot(n_values, list_catalan_def, 'x', label='Por definición (1)', color='blue')
plt.plot(n_values, list_catalan_rec, '.', label='Por recurrencia (2)', color='orange')

# Línea horizontal que representa la cota M
plt.axhline(y=M, color='red', linestyle='--', label=f'Cota M = {M}')

\end{minted}

Para el valor $M=10^{16}$, podemos observar el gráfico obtenido desde el codigo que hemos programado en la figura (\ref{fig:numeros-catalan}).
    
    \begin{figure}
        \centering
        \includegraphics[width=0.8\linewidth]{img/guide-1/numeros-catalan.png}
        \caption{Gráfico Números de Catalán bajo la cota $M = 10^{16}$}
        \label{fig:numeros-catalan}
    \end{figure}
%%%%%%%%%%%%%%%%%%%%%%%%%%%%%%%%%%%%%%%%%%%%%%%%%%%%%%%%%%%%%%%%%%%%%%%%%%%%%%%%%%%%%%%%%%%%%%%%%%%%%%%%%%%%%%%%%%%%%%%%%%%%%%%%%%%%%%%%%%%%

\section{Comportamiento asintótico}

Para $n \gg 1$, los números de Catalán tienen un comportamiento asintótico de la forma:

\begin{equation}\label{eq:approx}
C_n \approx \frac{4^n}{n^{3/2} \sqrt{\pi}}
\end{equation}

A continuación, comprobaremos gráficamente este comportamiento asintótico. Para ello, implementando \texttt{Python}, definimos una función que calcula la aproximación al \(n\)-ésimo número de Catalán según la ecuación~(\ref{eq:approx}). Se utiliza precisión \textbf{simple} mediante la función \texttt{float32} del paquete \texttt{numpy}.
\begin{minted}{python}
def catalan_def(n):
    catalan_n = np.float32 ( fact(2*n)/( fact(n+1)* fact(n) ) )
    return catalan_n
\end{minted}
Se define un rango de valores enteros, que representa los primeros 99 números naturales, y se almacena en la variable \texttt{n\_values}. Estos valores se manejan utilizando precisión \textbf{simple} (\texttt{float32}) .
\begin{minted}{python}
n_values = np.arange(1, 100, dtype=np.float32)
\end{minted}

Los números de Catalán (\texttt{Cn}) y su aproximación asintótica (\texttt{Cn\_aprox}) se calculan para cada valor en \texttt{n\_values} utilizando comprensiones de listas. Los resultados se almacenan como arreglos de \texttt{NumPy}.  
\begin{minted}{python}
Cn = np.array([catalan_def(i) for i in n_values])
Cn_aprox = np.array([catalan_aprox(i) for i in n_values])
\end{minted}

Es importante notar que, al usar precisión simple en las funciones \texttt{catalan\_def} y \texttt{catalan\_aprox}, se observan las siguientes limitaciones:  
\begin{itemize}
    \item Con \texttt{catalan\_def}, es posible calcular únicamente los primeros \(n = 68\) números de Catalán de forma explícita (excluyendo \(n = 0\)).

    \item Con \texttt{catalan\_aprox}, se pueden determinar los primeros $n = 62$ números de Catalán mediante la aproximación, excluyendo $n=0$.  
\end{itemize}
Los valores restantes en los arreglos \texttt{Cn} y \texttt{Cn\_aprox} serán representados como \texttt{inf}, lo que en \texttt{Python} indica infinito \cite{geeksforgeeks_infinity}.

Al generar un gráfico para comparar las curvas de los valores $C_n$ calculados por definición (\ref{ch:numeros-catalan}) y por la aproximación asintótica (\ref{eq:approx}), se obtiene la siguiente representación gráfica:

\begin{figure}[H]
    \centering
    \includegraphics[width=0.8\textwidth]{img/guide-1/numeros-catalan-aprox.png}
    \caption{Comparación gráfica entre $C_n$ y su aproximación asintótica.}
\end{figure}

Si se realiza un acercamiento ("zoom") entre las curvas, se puede observar con mayor claridad el comportamiento asintótico de la aproximación. Se aprecia esto en la figura \ref{fig:asintotic}.
\begin{figure}[H]
    \centering
    \includegraphics[width=0.8\textwidth]{img/guide-1/numeros-catalan-aprox-asintotic.png}
    \caption{Detalle gráfico del comportamiento asintótico entre $C_n$ y su aproximación.}
    \label{fig:asintotic}
\end{figure}
\section{Conclusión}

El estudio de los números de Catalán permitió validar su definición y recurrencia, explorar su comportamiento asintótico y analizar gráficamente sus propiedades, destacando su relevancia como una secuencia combinatoria fundamental mediante métodos numéricos y computacionales.

\end{document}
