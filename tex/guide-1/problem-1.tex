\documentclass[../portafolio.tex]{subfiles}

\begin{document}

\chapter{Simulación computacional de la caída libre}
\label{ch:caída-libre}
%%%%%%%%%%%%%%%%%%%%%%%%%%%%%%%%%%%%%%%%%%%%%%%%%%%%%%%%%%%%%%%%%%%%%%%%%%%%%%%%

\hfill \textbf{Fecha de la actividad:} 26 de octubre de 2024

\medskip

    En este capítulo, desarrollaremos una simulación computacional de la caída libre de una pelota desde una altura inicial $y_0 = H$, partiendo del reposo ($v_0 = 0$), bajo la acción de la aceleración gravitatoria $g$. Ignoraremos la resistencia del aire y otras fuerzas externas para simplificar el problema. Utilizaremos las ecuaciones del Movimiento Rectilíneo Uniformemente Acelerado (MRUA):  

    \begin{itemize}
        \item \textbf{Posición en función del tiempo}:
        \begin{equation}
            \label{eq:posicion-en-el-tiempo}
            y(t) = H - \frac{1}{2} \cdot g \cdot t^2
        \end{equation}
    
        \item \textbf{Velocidad en función del tiempo}:
        \begin{equation}
            \label{eq:velocidad-en-el-tiempo}
            v(t) = - g \cdot t
        \end{equation}
    \end{itemize}

    Donde $H > 0$ es la altura inicial, y $g = 9.81 \, \text{m/s}^2$ es la aceleración gravitatoria en la superficie terrestre. La simulación calculará las velocidades en intervalos regulares de tiempo $\Delta t$ mediante un ciclo \texttt{for}, e imprimirá los resultados en pantalla. Adicionalmente, se generará un gráfico de altura vs tiempo que destacará con una flecha el instante de impacto con el suelo.

\section{Deducción del instante de impacto $t_f$}

    Cuando la pelota toca el suelo, $y(t_f) = 0$. Evaluando este instante en la ecuación (\ref{eq:posicion-en-el-tiempo}), obtenemos:
        \begin{equation}
            \label{eq:deducción-tiempo-caida}
                0 = H - \frac{1}{2} \cdot g \cdot t_f^2
        \end{equation}
    Despejando para $t_f$:
    \begin{equation}
        \label{eq:tiempo-impacto}
        t_f = \sqrt{\frac{2 H}{g}}
    \end{equation}
    
    \section{Implementación de Python}
    
    Definimos los parámetros de simulación, estos son $H$ en metros, $g$ en metros sobre segundos cuadrados y $\Delta t$ segundos. Suponiendo $H = 100$ m, $g = 9.81 $ m/s $^2$ y $\Delta t = 0.01$ s. Así, tenemos en \texttt{Python}:
        \begin{minted}{python}
            H = 100  
            Delta t = 0.01
            g = 9.81
        \end{minted}
    
    Creamos una matriz de tiempos desde $t = 0$ hasta $t_f$, con $t_f > 0$:
        \begin{minted}{python}
            tiempo = np.arange(0, np.sqrt(2 * H / g), Dt)
        \end{minted}
        
    Calculamos e imprimimos la velocidad en cada instante, empleando la ecuación (\ref{eq:velocidad-en-el-tiempo}):
        \begin{minted}{python}
            for t in tiempo:
                velocidad = - g * t
                print(f"t = {t:.2f} s, v(t) = {velocidad:.2f} m/s")
        \end{minted}

\section{Gráfico Altura vs Tiempo}

    Guardamos espacio en la memoria para las posiciones correspondientes a cada instante, y las calculamos en un ciclo \texttt{for}:
        \begin{minted}{python}
            posiciones = np.zeros(len(tiempo))
            
            for i in range(len(tiempo)):
                posiciones[i] = H - 0.5 * g * tiempo[i]**2
        \end{minted}

    Generamos el gráfico con \texttt{matplotlib}:
        \begin{minted}{python}
            plt.plot(tiempo, posiciones, label="Altura (m) vs Tiempo (s)")
        \end{minted}
        
    Para el titulo del gráfico, queremos que indique el valor de H en el titulo, para ello usamos la siguiente linea de codigo:
        \begin{minted}{python}
            plt.title(f"Caída libre de una pelota desde H = {H} m respecto del suelo")
        \end{minted}     
    
    Empleando la función \texttt{annotate} de \text{matplotlib}, se inserta una flecha roja en el instante de impacto, junto a una etiqueta que indica el "Momento de impacto". Esta es acción es llamada con el código siguiente:
        \begin{minted}{python}
            plt.plot(tiempo, posiciones)
            plt.annotate('Momento de impacto', 
                         xy=(tiempo[-1], 0), 
                         xytext=(tiempo[-1]-1, H*(2/5)),
                         arrowprops=dict(facecolor='red', shrink=0.05))
        \end{minted}

    Podemos observar el gráfico obtenido en la figura (\ref{fig:altura-vs-tiempo}).
    
    \begin{figure}
        \centering
        \includegraphics[width=0.7\linewidth]{img/guide-1/problem-1/altura-vs-tiempo.png}
        \caption{Gráfico Altura vs Tiempo, con flecha roja indicando el momento de impacto}
        \label{fig:altura-vs-tiempo}
    \end{figure}
    
\section{Conclusión}

    En esta actividad utilizamos las ecuaciones del Movimiento Rectilíneo Uniformemente Acelerado (MRUA) para modelar y simular la caída libre de una pelota mediante programación en \texttt{Python}. Se trabajó con matrices de tiempo, las cuales fueron generadas con la función \texttt{linspace} de \texttt{numpy}, donde el final de este array fue el tiempo de impacto $tf=\sqrt{\frac{2H}{g}}$. Se empleó un ciclo \texttt{for} para calcular las velocidades y posiciones en intervalos regulares. Además, se exploró el uso de la librería \texttt{matplotlib} para visualizar los resultados a través de gráficos, esto en el uso de la función \texttt{annotate}, \cite{matplotlib}.

\end{document}