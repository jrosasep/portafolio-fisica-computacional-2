\documentclass[../portafolio.tex]{subfiles}

\begin{document}

\chapter{Análisis del Problema de Kepler y Solución Numérica con el Método del Salto de la Rana}
\label{ch:kepler-salto-rana}
%%%%%%%%%%%%%%%%%%%%%%%%%%%%%%%%%%%%%%%%%%%%%%%%%%%%%%%%%%%%%%%%%%%%%%%%%%%%%%%%

\hfill \textbf{Fecha de la actividad:} 24 de noviembre de 2024

\medskip

En este capitulo, analisaremos las trayectorias de un cometeta bajo la acción gravitacional del Sol. La ecuación de movimiento estará gobernada por la ley de gravitación universal. Normalizaremos esta ecuación y las condiciones iniciales. Implementando métodos numéricos para resolver la ecuación diferencial. Variaremos las condiciones iniciales y observaremos diferentes tipos de trayectorias para el cometa. Finalmente, nos concentremos en la trayectoria eliptica, calcularemos sus propiedades como elipse y verificaremos si en ella se cumplen las leyes de Kepler.

\section{Normalización de la ecuación}

La ecuación de movimiento para un cometa bajo la acción gravitacional del Sol está dada por:
\begin{equation}
    \vec{r}''(t) = -\frac{GM}{|\vec{r}|^3} \vec{r}, 
    \label{eq-km:kepler-motion}
\end{equation}
donde $\vec{r}$ es el vector posición del cometa, $G$ es la constante gravitacional y $M$ es la masa del Sol.

Las condiciones iniciales son:

\begin{equation}    \label{eq-km:condicion-inicial}
    \vec{r}(0) = r_0 \hat{x}, \quad \vec{r}'(0) = v_0 \hat{y}.
\end{equation}

A continuación se efectúa un \textbf{análisis dimensional} a cada variable en la ecuación \eqref{eq-km:kepler-motion}:

\begin{gather*}
    \left[t\right] = T , \quad
    \left[G\right] = \frac{L^3}{M \cdot T^2} , \quad
    \left[M\right] = M , \quad
    \left[\vec{r}\right] = L, \quad
    \left[ \vec{r}'' \right] = \frac{L}{T^2}
\end{gather*}

Notamos que \eqref{eq-km:kepler-motion} tiene \textbf{n} $= 5$ variables, donde cada una de las cuales puede expresarse en función de \textbf{k} $= 3$ magnitudes fundamentales: 
    \begin{itemize}
        \item longitud (L)
        \item tiempo (T)
        \item masa (M)
    \end{itemize}

Según el \textbf{Teorema de Pi}, la ecuación \eqref{eq-km:kepler-motion} puede reescribirse de forma equivalente como una ecuación que involucra \textbf{n - k} $= 2$ variables \textbf{normalizadas}, construidas a partir de las variables originales. Por lo tanto, introducimos las siguientes definiciones para las variables normalizadas de tiempo ($T$) y posición ($\vec{R}$):

\begin{equation}\label{eq-km:normalizaciones}
    T = t \sqrt{\frac{GM}{r_0^3}}, \quad \vec{R} = \frac{\vec{r}_i}{r_0} 
\end{equation}

Despejando $t$ y $\vec{r}$ a partir de sus respectivas expresiones en \eqref{eq-km:normalizaciones}, se obtiene:

\begin{equation} \label{eq-km:normalizaciones-despeje-2}
    t = T \sqrt{\frac{r_0^3}{G M}}, \quad \vec{r} = \vec{R} r_0
\end{equation}

A continuación, se realiza un cambio de variable en \eqref{eq-km:kepler-motion}, sustituyendo en dicha ecuación los valores de $t$ y $\vec{r}$ dados en \eqref{eq-km:normalizaciones-despeje-2}. Considerando $\vec{r}'' = \frac{d^2\vec{r}}{dt^2}$, tenemos que:

\begin{equation*}
    \frac{d^2(\vec{R} r_0)}{d\left(T \sqrt{\frac{r_0^3}{G M}}\right)^2} = -\frac{GM}{|\vec{R} r_0|^3} \vec{R} r_0
\end{equation*}

Luego de desarrollar y simplificar términos, se obtiene la normalización de la ecuación \eqref{eq-km:kepler-motion}:

\begin{equation}    \label{eq-km:kepler-motion-normalizado}
    \frac{d^2 \vec{R}}{dT^2} = -\frac{\vec{R}}{|\vec{R}|^3}
\end{equation}

Para \textbf{normalizar las condiciones iniciales}, se sustituyen en \eqref{eq-km:condicion-inicial} los valores de $t$ y $\vec{r}$ en \eqref{eq-km:normalizaciones-despeje-2}.
\begin{equation*}
    \vec{R}(0) r_0= r_0 \hat{x}, \quad \frac{d \vec{R}(0)r_0}{d\left(T \sqrt{\frac{r_0^3}{G M}}\right)} = v_0 \hat{y}.
\end{equation*}
Desarrollando estas ecuaciones, se obtienen las condiciones iniciales normalizadas:
\begin{equation}    \label{eq-km:condicion-inicial-normalizada}
    \vec{R}(0) = \hat{x}, \quad \frac{d \vec{R}(0)}{dT} = \tilde{v}_0\hat{y}.
\end{equation}

Donde $\tilde{v}_0 = v_0 \sqrt{\frac{r_0}{G M}}$ .

\section{Resolución numérica mediante el método del salto de la rana}

Empleamos el método del \textbf{salto de la rana} (\textit{leapfrog}) para resolver  numéricamente el problema de Kepler normalizado \eqref{eq-km:kepler-motion-normalizado}, y verificaremos la conservación de la energía y el momento angular en la solución del sistema . 

Para esto implementaremos un programa en \texttt{Python}, en el cual importamos las siguientes librerías:
\begin{itemize}
    \item \texttt{numpy}:  Para cálculos matemáticos eficientes.
    \item \texttt{matplotlib.pyplot}: Para graficar las trayectorias, energía total y momento angular.
    \item \texttt{EcuacionesDiferencialesOrdinarias}: Este módulo contiene la implementación del método del \textit{salto de la rana}, utilizado para resolver ecuaciones diferenciales acopladas. \cite{EcuacionesDiferencialesOrdinarias.} 
\end{itemize}

Definimos una función en \texttt{a\_kepler\_normalizado} que represente la aceleración normalizada del cuerpo, sabemos que esta es dada por la ecuación \eqref{eq-km:kepler-motion-normalizado}.  Esto de la forma:

\begin{minted}{python}
def a_kepler_normalizado(r, t):
    norm_r3 = np.linalg.norm(r) ** 3
    return -r / norm_r3
\end{minted}

Se establecen las condiciones iniciales normalizadas la de posición y velocidad para diferentes valores de $\tilde{v}_0$, según lo demostrado en \eqref{eq-km:condicion-inicial-normalizada}. Los tiempos se definen en un rango $[0, t_{\text{max}}]$ con un paso $\Delta t$ pequeño. Esto de forma:

\begin{minted}{python}
v0_tildes = [0.5, 1.0, 1.5] 
t_max = 20  
dt = 0.01   
t = np.arange(0, t_max, dt)

for v0_tilde in v0_tildes:
    # Condiciones iniciales normalizadas
    r0 = np.array([1.0, 0.0])  # Inicialmente en el eje x
    v0 = np.array([0.0, v0_tilde])  # Velocidad inicial en dirección y
\end{minted}

Utilizamos el método del \textit{salto de la rana}, implementado en la función \texttt{SaltoRana}, para resolver las ecuaciones acopladas de posición y velocidad. La posición y velocidad se calculan iterativamente.

\begin{minted}{python}
 r, v = SaltoRana(a_kepler_normalizado, r0, v0, t)

 \end{minted}

 Después de obtener las soluciones, verificamos la validez de la simulación numérica calculando en cada instante la Energía total y el Momento angular del sistema, esto según las ecuaciones:
\begin{itemize}
    \item \textbf{Energía total:}
    \begin{equation*}
        E = \frac{1}{2}\|\vec{V}\|^2 - \frac{1}{\|\vec{R}\|}
    \end{equation*}
    que incluye la energía cinética y potencial gravitacional.
    
    \item El \textbf{Momento angular:}
    \begin{equation*}
        L = \vec{R} \times \vec{V},
    \end{equation*}
    que es el producto cruz entre posición y velocidad.
\end{itemize}

Programamos esto de forma:
\begin{minted}{python}
    energia = 0.5 * np.sum(v**2, axis=1) - 1.0 / np.linalg.norm(r, axis=1)
    momento_angular = np.cross(r, v)
\end{minted}

Finalmente generamos tres gráficos:
        \begin{enumerate}
            \item Trayectoria (normalizada) en el plano $(x, y)$ (Figura \ref{fig-km:trayectoria}).
            \item Conservación de la energía total a lo largo del tiempo (Figura \ref{fig-kp:energia}).
            \item Conservación del momento angular a lo largo del tiempo (Figura \ref{fig-kp:momento}).
        \end{enumerate}
        Cada gráfico permite visualizar el impacto de distintos valores de $\tilde{v}_0$.

\begin{figure}[H]
    \centering
    \includegraphics[width=0.5\linewidth]{img//guide-4//kepler/kepler-trayectoria.png}
    \caption{Trayectoria del cometa (normalizada)}
    \label{fig-km:trayectoria}
\end{figure}

\begin{figure}[H]
    \centering
    \includegraphics[width=0.5\linewidth]{img//guide-4//kepler/keepler-energia.png}
    \caption{Conservación de la energía}
    \label{fig-kp:energia}
\end{figure}

\begin{figure}[H]
    \centering
    \includegraphics[width=0.5\linewidth]{img//guide-4//kepler/keepler-momento.png}
    \caption{Conservación del momento angular}
    \label{fig-kp:momento}
\end{figure}

\section{Implementación numérica}
A partir de los resultados obtenidos con el método del salto de la rana, se procede a:
\begin{itemize}
    \item Extraer $\vec{R}(t)$ y $\vec{V}(t)$ en intervalos de tiempo discretos.
    \item Determinar $R_{\text{max}}$ y $r_{\text{min}}$ directamente del conjunto de datos.
    \item Calcular los parámetros orbitales ($a$, $b$, $e$).
    \item Comparar las propiedades de la órbita obtenida con las leyes de Kepler.
\end{itemize}

\begin{minted}{python}
def calcular_parametros_orbitales(r):

    distancias = np.linalg.norm(r, axis=1)
    r_max = np.max(distancias)  # Distancia máxima
    r_min = np.min(distancias)  # Distancia mínima

    a = (r_max + r_min) / 2
    e = (r_max - r_min) / (r_max + r_min)

    b = a * np.sqrt(1 - e**2)

    return a, b, e

\end{minted}

\section{Análisis de la trayectoria elíptica}

En el gráfico \ref{fig-km:trayectoria}, vemos que para una velocidad inicial normalizada $\tilde{v} \cong 0.5$ el cometa tiene una trayectoria elíptica cerrada. Implementamos \texttt{python} para calcular los siguientes parámetros de esta trayectoria elíptica:

\begin{itemize}
    \item \textbf{Semieje mayor ($a$)}: El semieje mayor se calcula como el promedio de la distancia máxima ($r_{\text{max}}$) y mínima ($r_{\text{min}}$):
        \begin{equation}    
            a = \frac{R_{\text{max}} + R_{\text{min}}}{2}.
        \end{equation}
        Al calcular se obtuvo un semieje mayor $a=0.5718$.
    \item \textbf{Semieje menor ($b$)}: El semieje menor está relacionado con el semieje mayor y la excentricidad ($e$) mediante:
        \begin{equation}
           b = a \sqrt{1 - e^2}. 
        \end{equation}
        Al calcular se obtuvo un semieje menor $b= 0.3789$.
    \item \textbf{Excentricidad ($e$)}: La excentricidad se define como:
        \begin{equation}
            e = \frac{R_{\text{max}} - R_{\text{min}}}{R_{\text{max}} + R_{\text{min}}}.
        \end{equation}
        Al calcular se obtuvo una excentricidad $e=0.7490$
\end{itemize}


\section{Verificación de las leyes de Kepler}

Se verifican las tres leyes de Kepler \cite{kepler_laws} utilizando los resultados obtenidos a partir de la simulación numérica. 

\begin{itemize}
    \item \textbf{Primera ley de Kepler}: Para verificar esta ley, se comparan las posiciones $(x, y)$ calculadas numéricamente con la ecuación de una elipse:
    \begin{equation}
        \frac{x^2}{a^2} + \frac{y^2}{b^2} = 1,
    \end{equation}
    Al terminar la verificación se concluye que la primera ley no se cumple.

    \item \textbf{Segunda ley de Kepler}: Establece que el radio vector que une el cuerpo al centro de masas barre áreas iguales en intervalos de tiempo iguales. 

    Numéricamente, se calcula el área barrida en un intervalo de tiempo $\Delta t$ como:
    \begin{equation}
        \Delta A = \frac{1}{2} |\vec{r}(t) \times \vec{v}(t)| \Delta t,
    \end{equation}
    donde $\vec{R}(t)$ es la posición y $\vec{V}(t)$ la velocidad del cometa. Al terminar la verificación se concluye que la segunda ley si se cumple.

    \item \textbf{Tercera ley de Kepler}: Establece que el cuadrado del periodo orbital $T$ es proporcional al cubo del semieje mayor $a$:
    \begin{equation}    
        T^2 \propto a^3.
    \end{equation}
    El periodo orbital $T = 1.26$ se determina como el tiempo necesario para completar una revolución, y el semieje mayor fue calculado, obteniendo $a=0.5718$. Se calcula la relación $\frac{T^2}{a^3}$ y se obtiene un valor constante de $8.4934$. Por lo tanto se cumple la tercera ley de Kepler.
\end{itemize}

\section{Conclusión}

La normalización de la ecuación de movimiento del cometa, junto con las condiciones iniciales, nos proporciona un sistema más manejable desde el punto de vista numérico y conceptual. La ecuación resultante, que depende únicamente de parámetros adimensionales, es universal y aplicable a cualquier problema similar mediante un simple reescalado. Esto sienta las bases para su resolución numérica y para el análisis detallado de las trayectorias orbitales y su comparación con las leyes de Kepler.


\end{document}