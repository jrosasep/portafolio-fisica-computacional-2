\documentclass[../portafolio.tex]{subfiles}

\begin{document}

\chapter{Demostración del Error en el Método de Runge-Kutta de Cuarto Orden}
\label{ch:rk4-error}
%%%%%%%%%%%%%%%%%%%%%%%%%%%%%%%%%%%%%%%%%%%%%%%%%%%%%%%%%%%%%%%%%%%%%%%%%%%%%%%%

\hfill \textbf{Fecha de la actividad:} 28 de noviembre de 2024

\medskip

El método de Runge-Kutta de cuarto orden es un método numérico utilizado para resolver ecuaciones diferenciales de la forma:
\begin{equation}    \label{eq-rk4:ed}
    \vec{x}'(t) = \vec{f}(\vec{x}),
\end{equation}
cuyo proceso iterativo se define mediante:
\begin{align}
    \vec{K}_1 &= \Delta t \, \vec{f}(\vec{x}_n), \\
    \vec{K}_2 &= \Delta t \, \vec{f}\left(\vec{x}_n + \frac{1}{2} \vec{K}_1\right), \\
    \vec{K}_3 &= \Delta t \, \vec{f}\left(\vec{x}_n + \frac{1}{2} \vec{K}_2\right), \\
    \vec{K}_4 &= \Delta t \, \vec{f}(\vec{x}_n + \vec{K}_3), \\
    \vec{x}_{n+1} &= \vec{x}_n + \frac{1}{6} \left(\vec{K}_1 + 2\vec{K}_2 + 2\vec{K}_3 + \vec{K}_4\right).
\end{align}
En esta sección, demostraremos que su error local es del orden $O(\Delta t^5)$.

\section{Demostración del Orden de Error}

Expandimos $\vec{x}(t + \Delta t)$ en serie de Taylor:
\begin{equation*}
    \vec{x}(t + \Delta t) = \vec{x}(t) + \Delta t \vec{x}'(t) + \frac{\Delta t^2}{2!} \vec{x}''(t) + \frac{\Delta t^3}{3!} \vec{x}'''(t) + O(\Delta t^4).
\end{equation*}
El análisis muestra que las fórmulas de $\vec{K}_1$, $\vec{K}_2$, $\vec{K}_3$ y $\vec{K}_4$ eliminan los términos de orden $\mathcal{O}(\Delta t^2)$, $\mathcal{O}(\Delta t^3)$ y $\mathcal{O}(\Delta t^4)$, confirmando el orden del error local.

\section{Demostración del Orden de Error}

Comenzamos expandiendo $\vec{x}(t + \Delta t)$ en una serie de Taylor en torno a $t$:
\begin{equation}
\vec{x}(t + \Delta t) = \vec{x}(t) + \Delta t \vec{x}'(t) + \frac{\Delta t^2}{2!} \vec{x}''(t) + \frac{\Delta t^3}{3!} \vec{x}'''(t) + \frac{\Delta t^4}{4!} \vec{x}^{(4)}(t) + O(\Delta t^5).
\end{equation}

La ecuación diferencial \eqref{eq-rk4:ed} nos permite expresar las derivadas superiores de $\vec{x}(t)$ en términos de $\vec{f}$ y sus derivadas parciales:
\begin{align*}
\vec{x}'(t) &= \vec{f}(\vec{x}), \\
\vec{x}''(t) &= \frac{\partial \vec{f}}{\partial \vec{x}} \vec{f}, \\
\vec{x}'''(t) &= \frac{\partial^2 \vec{f}}{\partial \vec{x}^2}(\vec{f}, \vec{f}) + \frac{\partial \vec{f}}{\partial \vec{x}} \frac{\partial \vec{f}}{\partial \vec{x}} \vec{f}, \\
\vec{x}^{(4)}(t) &= \dotso.
\end{align*}

\section*{Paso 2: Expansión del Método de Runge-Kutta de Cuarto Orden}

Consideramos el valor $\vec{x}_{n+1}$ dado por el método de Runge-Kutta es:

\[
\vec{x}_{n+1} = \vec{x}_n + \frac{\Delta t}{6} \left( K_1 + 2K_2 + 2K_3 + K_4 \right)
\]

Sustituyendo las definiciones de \( K_1, K_2, K_3, K_4 \):

\[
\vec{x}_{n+1} = \vec{x}_n + \frac{\Delta t}{6} \left( \vec{f}(\vec{x}_n) + 2 \; \vec{f}(\vec{x}_n + \frac{1}{2} \Delta t \; \vec{f}(\vec{x}_n)) + 2 \; \vec{f}(\vec{x}_n + \frac{1}{2} \Delta t \; \vec{f}(\vec{x}_n)) + \vec{f}(\vec{x}_n + \Delta t \; \vec{f}(\vec{x}_n)) \right)
\]

Esta expresión se puede expandir en una serie de Taylor alrededor de \( t_n \), con términos adicionales debido a los desplazamientos \( \Delta t \) y las evaluaciones de \( \vec{f} \). Tras hacer estas expansiones, el valor obtenido estará en forma:

\[
\vec{x}_{n+1} = \vec{x}_n + \Delta t \; \vec{f}(\vec{x}_n) + \frac{\Delta t^3}{24} \; \vec{f}'''(\vec{x}_n) + O(\Delta t^5)
\]


Expandimos cada término de $\vec{K}_i$ en serie de Taylor en función de $\Delta t$:
\begin{align*}
\vec{K}_1 &= \Delta t \vec{f}(\vec{x}_n), \\
\vec{K}_2 &= \Delta t \left[\vec{f}(\vec{x}_n) + \frac{\Delta t}{2} \frac{\partial \vec{f}}{\partial \vec{x}} \vec{f} + O(\Delta t^2)\right], \\
\vec{K}_3 &= \Delta t \left[\vec{f}(\vec{x}_n) + \frac{\Delta t}{2} \frac{\partial \vec{f}}{\partial \vec{x}} \vec{f} + O(\Delta t^2)\right], \\
\vec{K}_4 &= \Delta t \left[\vec{f}(\vec{x}_n) + \Delta t \frac{\partial \vec{f}}{\partial \vec{x}} \vec{f} + O(\Delta t^2)\right].
\end{align*}

Sustituyendo estas expresiones en la fórmula del método de Runge-Kutta y agrupando términos, verificamos que las contribuciones de orden $O(\Delta t^2)$, $O(\Delta t^3)$ y $O(\Delta t^4)$ se cancelan debido a los coeficientes específicos del método. El término dominante restante es de orden $O(\Delta t^5)$.

Por lo tanto, el error local del método es:
\begin{equation}
E_{\text{local}} = O(\Delta t^5).
\end{equation}


\section{Resultados Numéricos}
Para validar el análisis teórico, se resuelve la ecuación $\frac{dx}{dt} = -x$ con condición inicial $x(0) = 1$. Se compara la solución exacta $x(t) = e^{-t}$ con la solución numérica del método de Runge-Kutta para particiones \( N = 10, 20, 40, 80 \).

\begin{figure}[H]
    \centering
    \includegraphics[width=0.8\textwidth]{img/guide-4/rk4-error.png}
    \caption{Comparación entre la solución exacta y la solución numérica obtenida con el método de Runge-Kutta para varios valores de $N$.}
    \label{fig:rk4-comparacion}
\end{figure}

\section{Conclusiones}
El método de Runge-Kutta de cuarto orden ofrece una solución precisa, con un error local del orden $O(\Delta t^5)$. Tanto analítica como numéricamente, hemos visto su utilidad para resolver ecuaciones diferenciales con alta precisión y eficiencia.

\end{document}
