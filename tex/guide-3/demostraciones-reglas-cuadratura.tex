\documentclass[../portafolio.tex]{subfiles}

\begin{document}

\chapter{Demostraciones de Reglas de Integración}
\label{ch:demostraciones-reglas-cuadratura}
%%%%%%%%%%%%%%%%%%%%%%%%%%%%%%%%%%%%%%%%%%%%%%%%%%%%%%%%%%%%%%%%%%%%%%%%%%%%%%%%

\hfill \textbf{Fecha de la actividad:} 2 de diciembre de 2024

\medskip



En este capítulo, se demuestran diversas reglas de integración numérica, incluyendo métodos de cuadratura abiertos, como las reglas Gaussianas, y métodos cerrados, como las reglas de Newton-Cotes. Estas demostraciones utilizan el método de coeficientes no determinados para garantizar la exactitud en polinomios de ciertos grados.


\section{Demostración con la regla de cuadratura cerrada de Newton-Cotes}
La regla de integración es:
\[
\int_{-\pi}^{\pi} f(x) \, dx = \frac{1}{2} \left[ \frac{1}{\pi^2} \Big(f(\pi) - f(-\pi)\Big) + \frac{16}{\pi^2} \Big(f\left(\frac{\pi}{2}\right) - f\left(-\frac{\pi}{2}\right)\Big) \right].
\]

\subsection*{Demostración:}
Supongamos que la integral se aproxima como:
\[
\int_{-\pi}^{\pi} f(x) \, dx \approx A f(\pi) + B f(-\pi) + C f\left(\frac{\pi}{2}\right) + D f\left(-\frac{\pi}{2}\right).
\]
Queremos determinar los coeficientes \(A\), \(B\), \(C\), \(D\) de forma que la regla sea exacta para polinomios de grado \(n \leq 3\).

\paragraph{1. Expansión de Taylor:}
Sea \(f(x)\) un polinomio de grado 3:
\[
f(x) = c_0 + c_1 x + c_2 x^2 + c_3 x^3.
\]

\paragraph{2. Sustitución en la regla de cuadratura:}
Sustituimos los valores de \(f(x)\) en \(\pm \pi\) y \(\pm \frac{\pi}{2}\):
\begin{align*}
f(\pi) &= c_0 + c_1 \pi + c_2 \pi^2 + c_3 \pi^3, \\
f(-\pi) &= c_0 - c_1 \pi + c_2 \pi^2 - c_3 \pi^3, \\
f\left(\frac{\pi}{2}\right) &= c_0 + c_1 \frac{\pi}{2} + c_2 \frac{\pi^2}{4} + c_3 \frac{\pi^3}{8}, \\
f\left(-\frac{\pi}{2}\right) &= c_0 - c_1 \frac{\pi}{2} + c_2 \frac{\pi^2}{4} - c_3 \frac{\pi^3}{8}.
\end{align*}

\paragraph{3. Integración exacta:}
Calculamos la integral exacta para cada término:
\[
\int_{-\pi}^{\pi} c_0 \, dx = 2\pi c_0, \quad
\int_{-\pi}^{\pi} c_1 x \, dx = 0, \quad
\int_{-\pi}^{\pi} c_2 x^2 \, dx = \frac{2}{3}\pi^3 c_2, \quad
\int_{-\pi}^{\pi} c_3 x^3 \, dx = 0.
\]

\paragraph{4. Igualación de coeficientes:}
Comparando los términos, determinamos los valores de \(A\), \(B\), \(C\), \(D\):
\[
A = B = \frac{1}{2\pi^2}, \quad C = -D = \frac{8}{\pi^2}.
\]

Por lo tanto, la regla queda demostrada.



\section{Demostración con la regla de cuadratura abierta de Newton-Cotes}
La regla de integración es:
\[
\int_{0}^{2h} f(x) \, dx = \frac{h}{15} \Big[ 7f(0) + 16f(h) + 7f(2h) \Big] + \frac{h^2}{15} \Big[f'(0) - f'(2h)\Big].
\]

\subsection*{Demostración:}
Partimos de la expansión de Taylor para \(f(x)\) alrededor de \(x = 0\):
\[
f(x) = f(0) + f'(0)x + \frac{f''(0)x^2}{2!} + \frac{f'''(0)x^3}{3!} + \cdots.
\]

\paragraph{1. Sustitución en los puntos:}
Evaluamos \(f(x)\) en \(x = 0\), \(x = h\), y \(x = 2h\):
\begin{align*}
f(0) &= f(0), \\
f(h) &= f(0) + f'(0)h + \frac{f''(0)h^2}{2} + \frac{f'''(0)h^3}{6}, \\
f(2h) &= f(0) + 2f'(0)h + 4\frac{f''(0)h^2}{2} + 8\frac{f'''(0)h^3}{6}.
\end{align*}

\paragraph{2. Sustitución en la regla:}
Sustituimos estos valores en la fórmula:
\[
\int_{0}^{2h} f(x) \, dx = \frac{h}{15} \Big[ 7f(0) + 16f(h) + 7f(2h) \Big] + \frac{h^2}{15} \Big[f'(0) - f'(2h)\Big].
\]

Verificamos que los términos coinciden con la expansión exacta de la integral.



\section{Demostración con la regla de regla de Gauss-Legendre}
La regla de integración es:
\[
\int_{-1}^{1} f(x) \, dx = \frac{1}{4} \Big[ 3f\left(-\frac{2}{3}\right) + 2f(0) + 3f\left(\frac{2}{3}\right) \Big].
\]

\subsection*{Demostración:}
Esta es una regla de Gauss-Legendre para \(n = 3\), por lo que los nodos y pesos se calculan para maximizar la precisión.

\paragraph{1. Nodos y pesos:}
Los nodos son:
\[
x_1 = -\frac{2}{3}, \quad x_2 = 0, \quad x_3 = \frac{2}{3}.
\]
Los pesos son:
\[
w_1 = w_3 = \frac{3}{4}, \quad w_2 = \frac{2}{4}.
\]

\paragraph{2. Verificación:}
La regla de Gauss-Legendre es exacta para polinomios de grado \(2n-1 = 5\). Sustituimos \(f(x) = 1\), \(f(x) = x\), \(f(x) = x^2\), etc., y comprobamos que ambos lados de la fórmula coinciden.

\section{Conclusión}

Hemos demostrado diversas reglas de integración numérica, incluyendo métodos abiertos y cerrados, utilizando el método de coeficientes no determinados. Estas reglas son exactas para polinomios de ciertos grados y proporcionan una base teórica para su aplicación en cálculos numéricos precisos.


\end{document}

