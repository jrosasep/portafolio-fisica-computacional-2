\documentclass[../portafolio.tex]{subfiles}

\begin{document}

\chapter{Deducción de los pesos de una regla exacta para polinomios de grado 3}
\label{ch:pesos-grado3}
%%%%%%%%%%%%%%%%%%%%%%%%%%%%%%%%%%%%%%%%%%%%%%%%%%%%%%%%%%%%%%%%%%%%%%%%%%%%%%%%

\hfill \textbf{Fecha de la actividad:} 1 de diciembre de 2024

\medskip

En este capítulo, se aborda el problema de encontrar una regla de integración que sea exacta para polinomios de hasta grado 3, basada en el conocimiento de $f(x)$ y su derivada $f'(x)$ en los extremos del intervalo $[0, 1]$. La integral a aproximar está dada por:
\begin{equation}
    \int_0^1 f(x) \, dx = w_0 f(0) + w_1 f(1) + w_2 f'(0) + w_3 f'(1),
\end{equation}

donde se deben determinar los pesos $w_0$, $w_1$, $w_2$, y $w_3$. Además, demostraremos que el error de este método depende de la cuarta derivada de $f(x)$, siendo proporcional a $h^5$, con $h = 1$.

\section{Cálculo de los pesos}

Consideremos $f(x)$ como un polinomio de grado 3:
\[
f(x) = a_0 + a_1 x + a_2 x^2 + a_3 x^3.
\]

La integral exacta es:
\[
\int_0^1 f(x) \, dx = \int_0^1 \left( a_0 + a_1 x + a_2 x^2 + a_3 x^3 \right) dx 
= a_0 + \frac{a_1}{2} + \frac{a_2}{3} + \frac{a_3}{4}.
\]

Evaluemos $f(x)$ y $f'(x)$ en $x = 0$ y $x = 1$:
\[
f(0) = a_0, \quad f'(0) = a_1, \quad f(1) = a_0 + a_1 + a_2 + a_3, \quad f'(1) = a_1 + 2a_2 + 3a_3.
\]

Sustituyendo estos valores en la regla de integración:
\[
\int_0^1 f(x) \, dx = w_0 f(0) + w_1 f(1) + w_2 f'(0) + w_3 f'(1).
\]

Expandiendo y agrupando por coeficientes:
\[
\int_0^1 f(x) \, dx = (w_0 + w_1)a_0 + (w_1 + w_2 + w_3)a_1 + (w_1 + 2w_3)a_2 + (w_1 + 3w_3)a_3.
\]

Igualamos término a término con la integral exacta:
\[
\begin{aligned}
w_0 + w_1 &= 1, \\
w_1 + w_2 + w_3 &= \frac{1}{2}, \\
w_1 + 2w_3 &= \frac{1}{3}, \\
w_1 + 3w_3 &= \frac{1}{4}.
\end{aligned}
\]

Resolviendo el sistema:
\[
\begin{aligned}
w_0 &= \frac{11}{24}, \\
w_1 &= \frac{1}{8}, \\
w_2 &= \frac{1}{3}, \\
w_3 &= -\frac{1}{6}.
\end{aligned}
\]

\section{Demostración del error}

Sea $R(f)$ el residuo de la regla para funciones generales $f(x)$. Utilizando la fórmula de Taylor, expandimos $f(x)$ alrededor de $x = \xi \in [0, 1]$:
\[
f(x) = f(\xi) + f'(\xi)(x - \xi) + \frac{f''(\xi)}{2!}(x - \xi)^2 + \frac{f^{(3)}(\xi)}{3!}(x - \xi)^3 + \frac{f^{(4)}(\xi)}{4!}(x - \xi)^4 + \cdots
\]

Integrando término a término en $[0, 1]$, los términos hasta grado 3 son exactos por construcción. El error se origina en el término de cuarto grado:
\[
R(f) = \int_0^1 \frac{f^{(4)}(\xi)}{4!} (x - \xi)^4 dx.
\]

Resolviendo esta integral para $x \in [0, 1]$, encontramos que:
\[
R(f) = \frac{f^{(4)}(\xi)}{720}.
\]

Por lo tanto, el error del método depende de la cuarta derivada de $f(x)$ como $E = \frac{f^{(4)}(\xi)}{720}$, lo cual verifica lo solicitado.

\section{Conclusión}

Hemos deducido los pesos $w_0, w_1, w_2, w_3$ de la regla de integración y demostrado que el error del método depende de la cuarta derivada de $f(x)$, siendo proporcional a $h^5$ con $h = 1$ en este caso.

\end{document}