% Formato de portafolio
% Documento adaptado de https://github.com/PlasmaPhysicsUdeC/FondecyTeX por Roberto Navarro <roberto.navarro@udec.cl>

% Este documento es modular, es decir, esta separado en varios
% archivos para facilitar su uso. Ver uso del paquete `subfiles`:
% https://www.overleaf.com/learn/latex/Multi-file_LaTeX_projects

\documentclass[twoside]{report}
\usepackage[table]{xcolor}
\usepackage[paper=letterpaper,left=3cm,right=3cm,top=3cm,bottom=2cm,headheight=24.1638pt, includefoot]{geometry}
%\usepackage[margin=1in,headheight=24.1638pt]{geometry}
%\usepackage[bookmarks=false]{hyperref}
\usepackage{fancyhdr}
\usepackage{tabularx}
\usepackage{multirow}
\usepackage[colorlinks]{hyperref}
\hypersetup{
    citecolor=cyan!80!black,
    urlcolor=cyan
}
\usepackage{hhline}
\usepackage{amsmath}
\usepackage{amsfonts}
\usepackage{enumitem}
\setenumerate{itemsep=-3pt,topsep=3pt}
\setdescription{itemsep=-3pt,topsep=3pt,leftmargin=!,labelwidth=4.5cm}
\usepackage{pgfgantt}
\usepackage{tikz}
\usepackage{graphicx}
\graphicspath{{./img/} {./tex/img/} {../img/} }

\usepackage[spanish]{babel}
\usepackage[utf8]{inputenc}
\usepackage[T1]{fontenc}
\spanishdecimal{.}

% % bibliografia: descomente estas dos lineas para usar estilo numerico, e.g. [1].
% \usepackage[square,numbers,sort&compress]{natbib}
% \bibliographystyle{apsrev4-1}

% bibliografia: descomente estas dos lineas para usar estilo autor-año, e.g. (Navarro, 2022).
\usepackage[authoryear]{natbib}
\bibliographystyle{aipauth4-1}


% Para incluir codigos python.
% Necesita opcion -shell-escape para compilar
\usepackage[cachedir=/tmp/minted-caches]{minted}

% Estilo para todo tipo de codigo
\setminted{breaklines, xleftmargin=\parindent, numbersep=5pt, bgcolor=lightgray!40, fontsize=\small}

% Estilo solo para python
\setminted[python]{linenos, texcomments, mathescape, texcomments}

% subfiles permite que el documento sea modular
\usepackage{subfiles}

%%% Comente/Descomente las siguientes lineas para cambiar la fuente del texto
% \usepackage{DejaVuSans}
% \renewcommand*\familydefault{\sfdefault}
% \usepackage{sansmath}
% \sansmath

% \numberwithin{equation}{chapter}

\pagestyle{fancy}
\renewcommand{\footrulewidth}{0.4pt}
\renewcommand{\headrulewidth}{0.4pt}
\fancyfoot{}
\fancyfoot[RE,RO]{\thepage}
\fancyfoot[LO,LE]{\textcolor[RGB]{127,127,127}{Portafolio - Física Computacional II (2024)}}

\fancyhead[LE,RO]{}
\definecolor{tcc}{RGB}{217,217,217} % Table cell color

\renewcommand\tabularxcolumn[1]{m{#1}}
\setlength{\arrayrulewidth}{0.5pt}
\renewcommand{\arraystretch}{2}

% \renewcommand{\thesection}{\alph{section})}
% \renewcommand{\thesubsection}{\alph{section}.\arabic{subsection}}

\begin{document}

% ASEGÚRESE DE COLOCAR SUS DATOS EN LA PORTADA
\subfile{tex/portada}


\tableofcontents


%%% con \subfile se incluyen archivos externos
\subfile{tex/introduccion}
\subfile{tex/presentacion-estudiante}

\part*{Evidencias de aprendizaje}  % cuerpo del portafolio
\addcontentsline{toc}{part}{Evidencias de aprendizaje}
\markboth{Evidencias de aprendizaje}{Evidencias de aprendizaje}

%%%%%%%%%%%%%%%%%%%%%%%%%%%%%%%%%%%%%%%%%%%%%%%%%%%%%%%%%%%%%%%%%%%%%%
% Agregue su trabajo a partir de acá
%%%%%%%%%%%%%%%%%%%%%%%%%%%%%%%%%%%%%%%%%%%%%%%%%%%%%%%%%%%%%%%%%%%%%%

%Guia 0
\subfile{tex/guide-0/derivada-estimacion} %0pt
\subfile{tex/guide-0/diferenciacion} %0pt
\subfile{tex/guide-0/integracion} %0pt
%Guia 1
\subfile{tex/guide-1/caida-libre} %1pts
\subfile{tex/guide-1/numeros-catalan}  %2pts
\subfile{tex/guide-1/potencial-electrostatico}  %1pt
\subfile{tex/guide-1/orbita-satelite}  %2pts

%Guia 2
\subfile{tex/guide-2/taylor-diferenciacion} %1pt
\subfile{tex/guide-2/derivacion-numerica} %1pt
\subfile{tex/guide-2/esquemas-errores-optimización} %2pts
\subfile{tex/guide-2/derivacion-ruido} %2pts

%Guia 3
\subfile{tex/guide-3/pesos-grado2} %1pts
\subfile{tex/guide-3/pesos-grado3} %2pts
\subfile{tex/guide-3/demostraciones-reglas-cuadratura} %2pts
\subfile{tex/guide-3/implementacion-regla-trapezoidal} %1pts
%Guia 4
\subfile{tex/guide-4/kepler-salto-rana} %2pts
\subfile{tex/guide-4/edo-euler} %1pt
\subfile{tex/guide-4/rk4-error} %2pts
\subfile{tex/guide-4/bessel} %3pts 

%Guia 5
\subfile{tex/guide-5/hermite} %1pt
\subfile{tex/guide-5/metodos-iterativos} %2pts
\subfile{tex/guide-5/soluciones-ecuaciones} %1pt

%%%%%%%%%%%%%%%%%%%%%%%%%%%%%%%%%%%%%%%%%%%%%%%%%%%%%%%%%%%%%%%%%%%%%%
% Conclusiones y referencias
\subfile{tex/conclusion-global}


%%%%%%%%%%%%%%%%%%%%%%%%%%%%%%%%%%%%%%%%%%%%%%%%%%%%%%%%%%%%%%%%%%%%%%
% Puede usar el capítulo de apéndice para agregar sus códigos completos si lo desea
\appendix

% lista de referencias guardadas en referencias.bib
\bibliography{referencias}

\end{document}